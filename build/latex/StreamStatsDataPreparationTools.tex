%% Generated by Sphinx.
\def\sphinxdocclass{report}
\documentclass[letterpaper,10pt,english]{sphinxmanual}
\ifdefined\pdfpxdimen
   \let\sphinxpxdimen\pdfpxdimen\else\newdimen\sphinxpxdimen
\fi \sphinxpxdimen=.75bp\relax

\PassOptionsToPackage{warn}{textcomp}
\usepackage[utf8]{inputenc}
\ifdefined\DeclareUnicodeCharacter
% support both utf8 and utf8x syntaxes
\edef\sphinxdqmaybe{\ifdefined\DeclareUnicodeCharacterAsOptional\string"\fi}
  \DeclareUnicodeCharacter{\sphinxdqmaybe00A0}{\nobreakspace}
  \DeclareUnicodeCharacter{\sphinxdqmaybe2500}{\sphinxunichar{2500}}
  \DeclareUnicodeCharacter{\sphinxdqmaybe2502}{\sphinxunichar{2502}}
  \DeclareUnicodeCharacter{\sphinxdqmaybe2514}{\sphinxunichar{2514}}
  \DeclareUnicodeCharacter{\sphinxdqmaybe251C}{\sphinxunichar{251C}}
  \DeclareUnicodeCharacter{\sphinxdqmaybe2572}{\textbackslash}
\fi
\usepackage{cmap}
\usepackage[T1]{fontenc}
\usepackage{amsmath,amssymb,amstext}
\usepackage{babel}
\usepackage{times}
\usepackage[Bjarne]{fncychap}
\usepackage{sphinx}

\fvset{fontsize=\small}
\usepackage{geometry}

% Include hyperref last.
\usepackage{hyperref}
% Fix anchor placement for figures with captions.
\usepackage{hypcap}% it must be loaded after hyperref.
% Set up styles of URL: it should be placed after hyperref.
\urlstyle{same}
\addto\captionsenglish{\renewcommand{\contentsname}{Contents:}}

\addto\captionsenglish{\renewcommand{\figurename}{Fig.\@ }}
\makeatletter
\def\fnum@figure{\figurename\thefigure{}}
\makeatother
\addto\captionsenglish{\renewcommand{\tablename}{Table }}
\makeatletter
\def\fnum@table{\tablename\thetable{}}
\makeatother
\addto\captionsenglish{\renewcommand{\literalblockname}{Listing}}

\addto\captionsenglish{\renewcommand{\literalblockcontinuedname}{continued from previous page}}
\addto\captionsenglish{\renewcommand{\literalblockcontinuesname}{continues on next page}}
\addto\captionsenglish{\renewcommand{\sphinxnonalphabeticalgroupname}{Non-alphabetical}}
\addto\captionsenglish{\renewcommand{\sphinxsymbolsname}{Symbols}}
\addto\captionsenglish{\renewcommand{\sphinxnumbersname}{Numbers}}

\addto\extrasenglish{\def\pageautorefname{page}}

\setcounter{tocdepth}{1}



\title{StreamStats Data Preparation Tools}
\date{Jan 22, 2020}
\release{4.0beta}
\author{}
\newcommand{\sphinxlogo}{\vbox{}}
\renewcommand{\releasename}{Release}
\makeindex
\begin{document}

\pagestyle{empty}
\sphinxmaketitle
\pagestyle{plain}
\sphinxtableofcontents
\pagestyle{normal}
\phantomsection\label{\detokenize{index::doc}}



\chapter{About}
\label{\detokenize{index:about}}
The StreamStats Data Preparation Tools are meant to aid in the processing of digital elevation models (DEM) and hydrography data for ingestion into the US Geological Survey’s Stream Stats application. The tools and the associated workflow examples can be used to prepare DEM and hydrography subsets for local Stream Stats folders, prepare those data for use in hydro-enforcement, hydro-enforce the digital elevation model, and processe the resulting flow accumulation and flow direction grids for use in the ArcHydro data model.


\chapter{Installation}
\label{\detokenize{index:installation}}
Clone these tools onto your machine using the git clone commands. Or download the the repository using the link in the upper right of the repositry page here: \sphinxstylestrong{*Link to repo*}

Once downloaded, the data preparation ESRI ArcGIS toolbox can be accessed from the ArcCatalog pane in ArcMap or navigated to from ArcPro. The toolbox is compatable with both ArcMap and ArcPro, except for the final processing step, which relies on ArcHydro and only works with ArcMap.

The ArcGIS toolbox is build from a set of Python libraries that can be called from the command line or a scripting environment to facilitate processing large volumes of data. Please refer to the documentation of the Python libraries and the test\_scripts folder for information and examples on the usage of the tools on the command line. The tools run fastest via ArcPro or Python 3, but can still be used with ArcMap and Python 2.


\chapter{Citation}
\label{\detokenize{index:citation}}
Please cite these tools and documentation as:

Barnhart, T.B., K., Kolb, A. Rae, P. Steeves, M. Smith, and P. McCarthy, Stream Stats Data Preparation Tools, USGS Software Release, \sphinxstylestrong{*Citation URL Here*}


\section{Data Preparation Workflow}
\label{\detokenize{workflow:data-preparation-workflow}}\label{\detokenize{workflow::doc}}

\subsection{Exercise 1: Hydro-Enforcement}
\label{\detokenize{ex_1:exercise-1-hydro-enforcement}}\label{\detokenize{ex_1::doc}}

\subsubsection{Document overview}
\label{\detokenize{ex_1:document-overview}}
Editor’s Note: This exercise is run on the data in the Exercise 1 workspace. These instructions were originally developed by Al Rea and Pete Steeves for trainings over the years. These were updated by Dave Stewart in 2011. This version annotated by Kitty Kolb and others in 2016, 2018, and 2019. Last edits made before archiving were 13 September 2019.


\paragraph{\sphinxstyleemphasis{Step 0: Preparation}}
\label{\detokenize{ex_1:step-0-preparation}}

\subparagraph{\sphinxstylestrong{Check Equipment}}
\label{\detokenize{ex_1:check-equipment}}
To run the StreamStats Toolbox you will need the following installed on your machine:
\begin{itemize}
\item {} 
ArcGIS version 10.3.x

\item {} 
ArcInfo Workstation

\item {} 
ArcHydro

\end{itemize}

Without these programs installed, and \sphinxstyleemphasis{these exact versions} installed, the toolbox will not work!


\subparagraph{\sphinxstylestrong{Obtain necessary data}}
\label{\detokenize{ex_1:obtain-necessary-data}}

\subparagraph{\sphinxstyleemphasis{File Structure}}
\label{\detokenize{ex_1:file-structure}}\begin{itemize}
\item {} 
Create four folders, all at the same level on your file directory

\item {} 
They should be named
\begin{itemize}
\item {} 
NED

\item {} 
WBD

\item {} 
NHD

\item {} 
archydro

\end{itemize}

\item {} 
Do not nest them or give them alternate names, or else the tools will not be able to find the files when they are needed

\end{itemize}


\subparagraph{\sphinxstyleemphasis{NED}}
\label{\detokenize{ex_1:ned}}\begin{itemize}
\item {} 
For this exercise, 4 tiles have already been downloaded from the USGS seamless server.

\item {} 
Important Notes: if you are using custom data instead of 3DEP
\begin{itemize}
\item {} 
Put the elevation grids here anyhow.

\item {} 
They do not have to be named the same as 3DEP grids.

\item {} 
If they are in their own projection instead of decimal degrees, that is okay.

\item {} 
The rasters need to be stored as rasters in grid or tif format, preferably grid

\item {} 
If you store it as a Raster Dataset in a geodatabase, it will not work

\end{itemize}

\end{itemize}


\subparagraph{\sphinxstyleemphasis{WBD}}
\label{\detokenize{ex_1:wbd}}
\sphinxstyleemphasis{Using Exercise Data}
\begin{itemize}
\item {} 
Use sample HUC data file that I tweaked to match NRCS WBD Data.

\end{itemize}

\sphinxstyleemphasis{Real-world data prep}
\begin{itemize}
\item {} 
If you are using official WBD for real-world (not test) data prep:
\begin{itemize}
\item {} 
Use the watershed boundaries stored in the most recent NHD downloads for your study area. ← This is the preferred option

\item {} 
Or, from the NRCS Data Gateway: \sphinxurl{https://datagateway.nrcs.usda.gov/}
\begin{itemize}
\item {} 
On the lower right-hand side, under “I Want To” click the link for “Order by state.”

\item {} 
Choose your state in the center of the page.

\item {} 
Then download the ‘12 digit Watershed Boundary Dataset 1:24,000.’

\item {} 
These are updated quarterly, however.

\end{itemize}

\end{itemize}

\item {} 
Important Notes: If you are using your own in-house derived local boundaries
\begin{itemize}
\item {} 
Save it as a shapefile to the WBD folder.

\item {} 
Make sure the fields names in your shapefile are the same as the field names for regular WBD.

\end{itemize}

\item {} 
Field Naming Conventions
\begin{itemize}
\item {} 
You should have at least two fields in your shapefile
\begin{itemize}
\item {} 
HUC\_8

\item {} 
HUC\_12

\end{itemize}

\item {} 
Note: Current naming conventions for the NHD geodatabases are “HUC8” and “HUC12” with no underscore.
\begin{itemize}
\item {} 
The StreamStats tool will autopopulate the field with “HUC\_8” and “HUC\_12”

\item {} 
But if yours are named other things, you just need to remember to navigate to the proper field name.

\end{itemize}

\item {} 
There can be other fields in the shapefile, but they are superfluous and may slow processing holding it all in memory

\end{itemize}

\end{itemize}

\begin{figure}[htbp]
\centering
\capstart

\noindent\sphinxincludegraphics{{wbdexample1}.png}
\caption{Figure: Screen capture of an NHD geodatabase with relevant fields indicated}\label{\detokenize{ex_1:id2}}\end{figure}

\begin{figure}[htbp]
\centering
\capstart

\noindent\sphinxincludegraphics{{tablehuc8}.png}
\caption{Figure: Sample table from a HUC-8 feature class}\label{\detokenize{ex_1:id3}}\end{figure}

\begin{figure}[htbp]
\centering
\capstart

\noindent\sphinxincludegraphics{{tablehuc12}.png}
\caption{Figure: Sample table from a HUC-12 feature class}\label{\detokenize{ex_1:id4}}\end{figure}
\begin{itemize}
\item {} 
To create a WBD shapefile:
\begin{itemize}
\item {} 
Using the official WBD
\begin{itemize}
\item {} 
Download the 4-digit NHD geodatabase from the USGS

\item {} 
Navigate to the WBD feature class in the catalog tree

\item {} 
If you are using HUC-12 boundaries as inwalls
\begin{itemize}
\item {} 
Export the HUC-12 feature class to a shapefile

\item {} 
Do this for each 4-digit geodatabase in your area of study

\item {} 
Merge the exported HUC-12 shapefiles

\item {} 
Delete HUC-12s that are downstream of your region of study

\end{itemize}

\item {} 
If you are taking a no-inwalls approach and using only HUC-8 boundaries (or HUC-10)
\begin{itemize}
\item {} 
Export the feature class to a shapefile

\item {} 
Do this for each 4-digit geodatabase in your area of study

\item {} 
Merge the exported HUC-8 or HUC-10 shapefiles
\begin{quote}
\begin{itemize}
\item {} 
Populate the HUC-8 and HUC-12 fields

\item {} 
Depending on which level of feature class you are using, there will be hydrologic unit fields populated as seen above:

\item {} 
You can delete all the fields except for the HUC8 and HUC12 fields

\item {} 
Add whichever field is not included in the feature class already

\item {} 
A quick way to populate the HUC-8 field is to use a Left String Query:

\begin{sphinxVerbatim}[commandchars=\\\{\}]
\PYG{n}{Left}\PYG{p}{(}\PYG{p}{[}\PYG{n}{HUC12}\PYG{p}{]}\PYG{p}{,}\PYG{l+m+mi}{8}\PYG{p}{)}
\end{sphinxVerbatim}

\end{itemize}

which strips out the first 8 characters of the HUC12 code, as seen below:
\end{quote}

\end{itemize}

\end{itemize}

\item {} 
If you are using in-house DIY watershed boundaries
\begin{itemize}
\item {} 
The process and naming conventions are the same for DIY local processing unit watershed boundaries

\item {} 
Run the feature classes through a topology check
\begin{itemize}
\item {} 
There should be no overlaps between features

\item {} 
There should be no gaps between features

\end{itemize}

\end{itemize}

\end{itemize}

\end{itemize}

\begin{figure}[htbp]
\centering
\capstart

\noindent\sphinxincludegraphics{{queryblock}.png}
\caption{Figure: Example of the query block in field calculator view}\label{\detokenize{ex_1:id5}}\end{figure}


\subparagraph{\sphinxstyleemphasis{NHD}}
\label{\detokenize{ex_1:nhd}}\begin{itemize}
\item {} 
From NHD prestaged 4digit in FileGDB format from the Amazon Cloud site: \sphinxurl{https://prd-tnm.s3.amazonaws.com/index.html?prefix=StagedProducts/Hydrography/NHD/} download by HUC of interest

\item {} 
Important Notes: If you are using custom data instead of NHD out of the box
\begin{itemize}
\item {} 
Put the NHD here anyhow, so the Toolbox will be happy.
\begin{itemize}
\item {} 
You can paste your other data into the shell later.

\item {} 
It is better if your custom data is in an NHD-like format.

\end{itemize}

\item {} 
The new NHD gdb file format is named differently than the old version that the program expects. Make sure your NHD is renamed to look like this:

\end{itemize}

\end{itemize}

\begin{figure}[htbp]
\centering
\capstart

\noindent\sphinxincludegraphics{{nhdnaming}.png}
\caption{Figure: Example of properly formatted NHD geodatabases with the Format NHD plus HUC 4 code}\label{\detokenize{ex_1:id6}}\end{figure}

\sphinxstyleemphasis{NOTE:}
If you are using non-standard data, such as in-house LiDAR or in-house huc-boundaries, save them in same folder setup and rename them to be the same folder or file names as listed in the exercises. It is important to keep all input files together and not stored in various places on a hard drive. Otherwise the program will not be able to find the files. Many folder paths are hard-coded into the toolbox


\paragraph{\sphinxstyleemphasis{Step 1: Setup ArcHydro database}}
\label{\detokenize{ex_1:step-1-setup-archydro-database}}

\subparagraph{\sphinxstylestrong{Step 1a: creating the containing folders}}
\label{\detokenize{ex_1:step-1a-creating-the-containing-folders}}\begin{enumerate}
\def\theenumi{\arabic{enumi}}
\def\labelenumi{\theenumi .}
\makeatletter\def\p@enumii{\p@enumi \theenumi .}\makeatother
\item {} 
If you haven’t already, create an output folder in ArcCatalog called ‘archydro’ that will house all of the resultant exercise data.

\item {} 
In ArcCatalog add the StreamStats v3.10 toolbox and open the toolset ‘Setup tools’ and double click ‘Database Setup’

\item {} 
Set the Output Workspace name to the ‘archydro’ folder set above

\item {} 
For ‘Main ArcHydro Geodatabase Name’ type in ‘source\_data’ for this new file geodatabase, which will be used to house the huc feature class (generated here) and the raster catalog (generated in the ‘Second Step, NED Tools’)

\item {} 
For ‘WBD Dataset’ select the ‘sample\_wbd’ shapefile (under ‘WBD’ in the ‘exercise\_data’ folder) which in this case is already projected in the final state projection of choice (otherwise the ‘WBD Dataset is already projected’ button would need to be unchecked and the ‘WBD Projection Template if unprojected’ optional menu item would need to be populated with a feature dataset or shapefile that is already in the proper projection).

\item {} 
Be sure the HUC\_8 and HUC\_12 fields are set correctly for your dataset

\item {} 
The ‘HUC Buffer Distance (m)’ should be set to 2000 (meters).  This buffer distance is used for TopoGrid processing which is trimmed back down to the HUC boundary in subsequent steps.

\item {} 
Select the path to the NHD 4-digit High Resolution data (select ‘NHD’ under exercise\_data).  You must have already downloaded all 4-digit hydro regions covering your study area

\item {} 
Select one of your unprojected NED datasets to use as a source template for ‘NED Projection Template’ (under exercise\_dataNED, enter one of the four folders and select the grid)

\item {} 
Run the script.

\end{enumerate}

\begin{figure}[htbp]
\centering
\capstart

\noindent\sphinxincludegraphics{{databasesetup}.png}
\caption{Figure: view of the Database Setup script inputs}\label{\detokenize{ex_1:id7}}\end{figure}
\begin{enumerate}
\def\theenumi{\arabic{enumi}}
\def\labelenumi{\theenumi .}
\makeatletter\def\p@enumii{\p@enumi \theenumi .}\makeatother
\setcounter{enumi}{10}
\item {} 
Sometimes when using custom data (such as lidar-derived streams), Kitty has found that the Database Setup step will choke on the NHD portion, giving an error about “missing M values.” It is possible that M values are missing, or it is possible this is a red herring. In that case, use the “Database Setup No M Values” tool instead.

\end{enumerate}


\subparagraph{\sphinxstylestrong{Step 1b: Examine the outputs}}
\label{\detokenize{ex_1:step-1b-examine-the-outputs}}
After running, open ArcMap.  Under the ‘archydro’ folder, examine the feature class outputs for one of the newly generated ‘input\_data’ file geodatabases in one of the newly generated local workspaces (note the local workspaces have the name of the 8-digit HUC).  Also examine the ‘huc8index’ feature class in the new ‘source\_data’ file geodatabase (also in the ‘archydro’ folder).  Compare all data to the original NED, NHD and WBD datasets.  Several observations:
\begin{itemize}
\item {} 
‘NHDFlowline’ and ‘NHDFlowline\_orig’ are identical line feature classes. ‘NHDFlowline’ however, will be modified to your satisfaction as the input dataset for HydroDEM ‘burning’

\item {} 
‘huc8’ and ‘huc12’ datasets appear identical (the attribute tables are different).  Typically this would not be the case.  However for this exercise we are using the source huc12 features as a surrogate huc8 datasets to cut down on processing time. ‘huc8\_buffer\_dd83’ is used as input for the ‘Extract Polygon Area From NED’ tool which is used below (under NED Tools).

\item {} 
inwall\_edit’ should have all interior WBD boundaries that will be used for ‘walling’.
\begin{itemize}
\item {} 
For the sample dataset, there is nothing for the same reason as explained in the previous bullet.

\item {} 
This dataset could be modified at this time to include more interior boundaries such as gage boundaries

\item {} 
If you are not using inwalls in your data prep:
\begin{itemize}
\item {} 
Start an edit session on the “inwall\_edit” feature class

\item {} 
Select all features

\item {} 
Delete all features

\item {} 
Save the feature class

\end{itemize}

\end{itemize}

\item {} 
‘huc8\_buffer’ includes a 2000 meter buffer area which will be used in TopoGrid to clip the DEM data.  These buffer areas overlap.

\item {} 
‘NHDArea’ and ‘NHDWaterbody’ are used as input into bathymetric gradient processing.

\end{itemize}


\subparagraph{\sphinxstylestrong{Step 1c: WBD Intersect tools}}
\label{\detokenize{ex_1:step-1c-wbd-intersect-tools}}
\sphinxstyleemphasis{Note:} the WBD Intersect tools have not been updated in 9 years, and we no longer support the WBD Intersect tools.


\paragraph{\sphinxstyleemphasis{Step 2: NED Tools}}
\label{\detokenize{ex_1:step-2-ned-tools}}

\subparagraph{\sphinxstylestrong{Step 2a: Make NED Index}}
\label{\detokenize{ex_1:step-2a-make-ned-index}}\begin{enumerate}
\def\theenumi{\arabic{enumi}}
\def\labelenumi{\theenumi .}
\makeatletter\def\p@enumii{\p@enumi \theenumi .}\makeatother
\item {} 
In ArcCat, Run the first ‘NED Tools’ tool: ‘A. Make NED Index’

\item {} 
‘Output Geodatabase’:  Select the ‘source\_data’ file geodatabase you created with the ‘Database Setup’ tool

\item {} 
‘Output Raster Catalog Name’: Keep the default (IndexRC)

\item {} 
‘Coordinate System’: Import the source coordinate system (GCS\_North\_American\_1983) using one of the source grids

\item {} 
‘Input NED Workspace’: Select the NED folder

\item {} 
‘Output Polygon Feature Class’: Keep the default (IndexPolys)

\item {} 
Run the script

\end{enumerate}

\begin{figure}[htbp]
\centering
\capstart

\noindent\sphinxincludegraphics{{makenedindex}.png}
\caption{Figure: Sample inputs for the Make NED Index Tool}\label{\detokenize{ex_1:id8}}\end{figure}

After running, load and view the output in ArcMap (IndexPolys, IndexRC).  View the attribute tables.  Quit out of Arcmap without saving.


\subparagraph{\sphinxstylestrong{Step 2b: Extract Polygon Area From NED}}
\label{\detokenize{ex_1:step-2b-extract-polygon-area-from-ned}}\begin{enumerate}
\def\theenumi{\arabic{enumi}}
\def\labelenumi{\theenumi .}
\makeatletter\def\p@enumii{\p@enumi \theenumi .}\makeatother
\setcounter{enumi}{7}
\item {} 
In ArcCat, Run the 2nd tool in the ‘NED Tools’: ‘B. Extract Polygon Area From NED’, which is run on each ‘local’ workspace in the ‘archydro’ folder.

\item {} 
‘Output Workspace’: ‘01091111’ (in the ‘archydro’ folder)

\item {} 
‘NED Index Polygons’: Select ‘NEDIndexPolys’ in the ‘source\_data’ file geodatabase

\item {} 
‘Clip Polygon’: Select ‘huc8\_buffer2000\_dd83’ in the archydro01091111input\_data file geodatabase

\item {} 
‘Output Grid’: Keep the default (‘dem\_dd’)

\end{enumerate}

\begin{figure}[htbp]
\centering
\capstart

\noindent\sphinxincludegraphics{{extpolyarea}.png}
\caption{Figure: Sample of the inputs for the Extract Polygon Area tool}\label{\detokenize{ex_1:id9}}\end{figure}

(THIS TOOL CAN BE BATCHED)

Repeat the ‘Extract Polygon Area From NED’ on the other 3 ‘local’ workspaces (01092222, 01093333 and 01094444)

After running, load and view one of the ‘local’ folder output ‘dem\_dd’ grids in ArcMap (note the inclusion of the 2000 meter buffer area, which again, is needed for TopoGrid). Quit out of Arcmap without saving.


\subparagraph{\sphinxstylestrong{Steps 2c \& 2d: Check NoData \& Fill NoData}}
\label{\detokenize{ex_1:steps-2c-2d-check-nodata-fill-nodata}}
Run the 3rd tool in the ‘NED Tools’: \sphinxstyleemphasis{‘C. CheckNodata’}
\begin{enumerate}
\def\theenumi{\arabic{enumi}}
\def\labelenumi{\theenumi .}
\makeatletter\def\p@enumii{\p@enumi \theenumi .}\makeatother
\item {} 
InGrid: ‘dem\_dd’ (in 01091111)

\item {} 
OutPolys.shp: ‘NoDataChk’ (put in 01091111).  The output (default = C:tmpRasterT\_Singleo\textless{}value\_increment\textgreater{}.shp) should show up as a donut hole polygon area with no ‘GRIDCODE’ values of 1 showing up within.  If there were NODATA polys then you would either run the next tool, \sphinxstyleemphasis{‘Fill NODATA Cells’} (which replaces NODATA values in a grid with mean values within a 3x3 window) or acquire better NED data if available.

\item {} 
Repeat CheckNodata on 01092222, 01093333, and 01094444

\end{enumerate}

\begin{figure}[htbp]
\centering
\capstart

\noindent\sphinxincludegraphics{{checknodata}.png}
\caption{Figure: Sample inputs for the Check No Data tool}\label{\detokenize{ex_1:id10}}\end{figure}
\begin{enumerate}
\def\theenumi{\arabic{enumi}}
\def\labelenumi{\theenumi .}
\makeatletter\def\p@enumii{\p@enumi \theenumi .}\makeatother
\setcounter{enumi}{3}
\item {} 
It is possible you may find encapsulated areas of NoData on the edge of the buffer. These are not a problem as long as they are on the outside of the buffer. If you find NoData within the buffer, that is a problem. See illustration below:

\end{enumerate}

\begin{figure}[htbp]
\centering
\capstart

\noindent\sphinxincludegraphics{{nodatainsouts}.jpg}
\caption{Figure: Sample areas where NoData cells should be filled or not.}\label{\detokenize{ex_1:id11}}\end{figure}


\subparagraph{\sphinxstylestrong{Step 2e: Project and Scale NED}}
\label{\detokenize{ex_1:step-2e-project-and-scale-ned}}
In ArcCat Run the 5th tool in the ‘NED Tools’: ‘\sphinxstyleemphasis{E. Project and Scale NED}’

(this tool also sets a rounded origin coordinate for the output grid)
\begin{enumerate}
\def\theenumi{\arabic{enumi}}
\def\labelenumi{\theenumi .}
\makeatletter\def\p@enumii{\p@enumi \theenumi .}\makeatother
\item {} 
Input Workspace: 01091111

\item {} 
Input Grid: Keep the default (dem\_dd)

\item {} 
Output Grid: Keep the default (dem\_raw)

\item {} 
Output Coordinate System: Import the coordinate system for huc8index in the ‘source\_data’ file geodatabase in the archydro root folder.  The coordinate system is USA Albers USGS.

\item {} 
Output Cell Size: 10 (the projection is Albers, meters)

\item {} \begin{description}
\item[{Registration Point:}] \leavevmode\begin{itemize}
\item {} 
Original text: For all StreamStats projects, this should be left alone (15 15)

\item {} 
Updated instructions: use Registration point 0 0 unless you really know what you are doing. It will default to 15 15- change this to 0 0.

\item {} 
Historical note: Al’s instructions are to make the registration point 15 15, because he wanted to make sure that the grids align with the NLCD. Kitty has found that using a registration point of 15 15 causes the wb\_srcg and nhd\_wbg grids to be shifted over half a cell. This has caused problems in Alaska and Pennsylvania. Kitty used 0 0 in NC, and everything worked out fine. The key is to be consistent throughout your state.

\item {} 
You really only want to use registration point 15 15 if you are doing an NHDPlus implementation

\end{itemize}

\end{description}

\end{enumerate}

\begin{figure}[htbp]
\centering
\capstart

\noindent\sphinxincludegraphics{{projectscalened}.png}
\caption{Figure: Sample inputs for the Project and Scale NED tool}\label{\detokenize{ex_1:id12}}\end{figure}

(THIS TOOL CAN BE BATCHED)

Repeat ‘Project and Scale NED’ for 01092222, 01093333, and 01094444.  Note that the output zunits are in integer centimeters (the input was in floating point meters). Also note that the output ‘dem\_raw’ grid includes a line in the projection file ‘ZUNITS 100’.  This custom (non-default) value is needed by several basin characteristics, including basin slope, since the Z value (centimeters) is different from the XY values (meters).  Otherwise basin slope would be exaggerated 100 times.


\paragraph{\sphinxstyleemphasis{Step 3: Editing NHD}}
\label{\detokenize{ex_1:step-3-editing-nhd}}

\subparagraph{\sphinxstylestrong{Preliminary Notes}}
\label{\detokenize{ex_1:preliminary-notes}}

\subparagraph{\sphinxstyleemphasis{Note about BYOD}}
\label{\detokenize{ex_1:note-about-byod}}
NOTE: If you are bringing your own data to the tools and not using the NHDFlowlines, this is the step where you delete the NHD-derived Flowlines and substitute your own stream lines in the “input\_data” geodatabase. (See further below for instructions.) Make sure you name your feature class “NHDFlowlines,” however.


\subparagraph{\sphinxstyleemphasis{Note about Editing Instructions}}
\label{\detokenize{ex_1:note-about-editing-instructions}}
The instructions that follow are a summarized overview of the editing process on data that originated in the NHD and are mostly clean. If you are working on your own data, or would like instructions that cover almost any contingency in excruciating detail with screen-captured examples from the South Carolina LiDAR data prep, you may wish to consult the document “SC StreamStats Linework Review Workflow”.


\subparagraph{\sphinxstylestrong{Exercise Instructions}}
\label{\detokenize{ex_1:exercise-instructions}}\begin{itemize}
\item {} 
Open ArcMap.

\item {} 
Navigate to the 01091111 folder and add the features in the ‘input\_data’ file geodatabase.

\item {} 
Turn all layers off except ‘huc8’ and ‘NHDFlowline’.

\item {} 
Under ‘File’ select ‘Add Data from ArcGIS online’.
\begin{itemize}
\item {} 
Load the ‘US Topo Maps’ and ‘World Imagery’ web services to your view.

\item {} 
Turn these layers off for now (they slow things down).

\end{itemize}

\item {} 
Observe the stream overlapping the Northern boundary of the huc.
\begin{itemize}
\item {} 
This is problematic.

\item {} 
Although the streamline represents flow in a wetland, the wetland straddles the boundary and flows in 2 directions (entering and exiting the huc).

\item {} 
Zoom into this general area and turn on the topo image.
\begin{itemize}
\item {} 
There is another issue. A small disconnect just South of the letter ‘D’ in ‘Dead Swamp’.

\item {} 
This gap is so narrow, if HydroDEM were to be run, the resulting flow accumulation would hop across the gap.

\item {} 
There are many other reasons to clean up an NHD dataset, but for this exercise, we need to address this one now.

\end{itemize}

\item {} 
Start an edit session on the ‘input\_data’ file geodatabase.
\begin{itemize}
\item {} 
Choose ‘NHDFlowline’ as your target layer.

\item {} 
Select the flowline in the swamp and delete it.  Save your edit.

\end{itemize}

\end{itemize}

\item {} 
Turn the topo image off and zoom back out to the extent of the huc8.
\begin{itemize}
\item {} 
The flow in the basin is generally North to South, with the outlet at the Southern- most point of the HUC.

\item {} 
Note near the outlet, there is a disconnect in the flow.

\item {} 
Zoom to this location and bring up the World imagery.

\item {} 
This disconnect is likely an underground conduit connecting the reservoir to the stream on the South side of the major road.

\end{itemize}

\item {} 
Let’s correct the problem.
\begin{itemize}
\item {} 
The target layer remains ‘NHDFlowline

\item {} 
Set ‘NHDFlowline’ Snapping to ‘End’

\item {} 
Digitize in a connector line in the direction of flow (North/South)

\item {} 
Save Edits

\end{itemize}

\item {} 
Remove all layers from the Table of Contents

\item {} 
Add all layers in the local 0109444 input\_data gdb

\item {} 
Turn all layers off and turn back on ‘huc8’ and ‘NHDFlowline’

\item {} 
Zoom to ‘huc8’
\begin{itemize}
\item {} 
This is the receiving HUC for the other 3 local HUCs.

\item {} 
Observe the 3 locations where those 3 upstream HUCs flow into this one.

\item {} 
Zoom into one of these locations

\end{itemize}

\item {} 
The ‘NHDFlowline’ feature class is eventually used in the burning process.
\begin{itemize}
\item {} 
Any feature in this dataset that is an inlet for a downstream HUC needs to be trimmed just inside the HUC.

\item {} 
Otherwise, the ‘walling’ process can get compromised at the boundary edge.

\item {} 
This trimming should only be a cell or 2 in length (10 \textendash{} 20 meters).

\end{itemize}

\item {} 
In the Editor toolbar, choose  Start Editing
\begin{itemize}
\item {} 
Target: ‘NHDFlowline’

\item {} 
Select the feature that overlaps the boundary.

\item {} 
Measure a distance

\item {} 
Split the feature

\item {} 
Delete the segment

\item {} 
Repeat this for the other 2 inlet features

\item {} 
Save and Stop editing

\item {} 
Quit out of ArcMap without saving

\end{itemize}

\end{itemize}

Note: There will typically be other edits to review for both the NHD and WBD.  This is why we isolate an edit-able copy for both (‘NHDFlowline’ and ‘inwall\_edit’). Also, for ‘NHDFlowline’ it may be helpful to occasionally reestablish a geometric network to check flow connectivity.  The best tools to do this are typically NHD tools or the ArcGIS Utility Network tools, but to use the NHD tools,  the feature class name typically needs to be ‘NHDFlowline’ and this dataset typically needs to be in a ‘Hydrography’ feature dataset.  Some manipulation of the ‘NHDFlowline’ feature class (including temporarily loading into a separate geodatabase) may be necessary. Instructions on the NHD Tools can be found here: (outdated link)


\subparagraph{\sphinxstylestrong{BYOD Streamline Editing Instructions}}
\label{\detokenize{ex_1:byod-streamline-editing-instructions}}\begin{itemize}
\item {} 
If you are bringing local data:
\begin{itemize}
\item {} 
Buffer your local processing unit (LPU, usually a huc8, but can be something else) boundary by X amount (usually 2 kilometers/2000 m)

\item {} 
Clip out the local data-equivalent feature classes to the NHDFlowline, NHDArea, and NHDWaterbody using the buffer

\item {} 
Add your local data equivalent buffer-clipped feature classes to your table of contents for the project

\item {} 
Start an edit session on the LPU input\_data.gdb.
\begin{itemize}
\item {} 
Open the table for NHDFlowline (not NHDFlowline\_orig)

\item {} 
Select all records in input\_dataNHDFlowline

\item {} 
Delete the selected records.

\item {} 
Save the edit session, but do not stop the edit session

\end{itemize}

\item {} 
Select all the records in the local equivalent NHDFlowline\_clipped\_buffer feature class

\item {} 
On the File status bar, got to Edit \(\rightarrow\) Copy (do not do “copy selected records”, this will not work)

\item {} 
Go to Edit \(\rightarrow\) Paste
\begin{itemize}
\item {} 
A window will open saying “Paste selected records into”

\item {} 
Choose the newly-empty NHDFlowline feature class

\item {} 
Click okay.

\end{itemize}

\item {} 
Save the edit session, stop the edit session

\end{itemize}

\item {} 
Follow the editing instructions as above
\begin{itemize}
\item {} 
Remove braids
\begin{itemize}
\item {} 
Use the Feature to Polygon tool to find areas that are braids

\item {} 
Zoom to each polygon to snip segments as necessary

\item {} 
When in doubt:
\begin{itemize}
\item {} 
Follow the named stream/river

\item {} 
Unless it is clearly much more flow on the map, and then you should contact the NHD stewardship team to get the named segment moved to another channel.

\end{itemize}

\end{itemize}

\item {} 
Do a topology check to make sure there are not two flowlines on top of each other

\item {} 
Do a geometric network check to make sure there are no disconnected segments and that streamlines are flowing in the correct direction
\begin{itemize}
\item {} 
Flip streamline segments as needed

\item {} 
Connect disconnected segments as needed

\item {} 
You will never get 100 \%, but do the best you can

\end{itemize}

\item {} 
Recheck for loops/braids accidentally created by connecting segments.

\end{itemize}

\end{itemize}


\paragraph{\sphinxstyleemphasis{Step 4: NHD-WBD Conflicts}}
\label{\detokenize{ex_1:step-4-nhd-wbd-conflicts}}

\subparagraph{\sphinxstylestrong{Overview}}
\label{\detokenize{ex_1:overview}}
The StreamStats Tools break the WBD shapefile into two polygon feature classes: huc12 and huc8. The huc12 features and/or streamgage basin boundaries are often used as the inner walls in the “walling” process. The huc8 polygons are used as the outer walls. Each huc12 and huc8 polygon should only have a single outlet.

Occasionally, the stream lines and the WBD lines intersect in places they shouldn’t or the 2 lines come very close to each other and could create an unwanted breach in an inner or outer wall.  Locations where the 2 feature classes intersect should be identified. They need to be visually examined and sometimes corrected. Most of the time, the WBD boundary is adjusted. Sometimes, the stream can be shifted slightly or snipped. This falls into the “Know your data” category of judgement calls; resources such as imagery, LiDAR breaklines, and local knowledge can help you determine which one to remove. If you are changing the WBD, in ideal circumstances you should also coordinate with the WBD stewards to make sure that your changes are in concert with them. This may not always be feasible due to time-constraints.

Here is an example of an area that needs close inspection:

\begin{figure}[htbp]
\centering
\capstart

\noindent\sphinxincludegraphics{{laboundarycross}.png}
\caption{Figure: Example of a place in Louisiana where a stream crosses a HUC boundary}\label{\detokenize{ex_1:id13}}\end{figure}


\subparagraph{\sphinxstylestrong{Workflow}}
\label{\detokenize{ex_1:workflow}}\begin{itemize}
\item {} 
Remove clearly outside streamlines
\begin{itemize}
\item {} 
Start an edit session

\item {} 
Select all features in NHDFlowline that intersect the local processing unit (LPU) feature class (probably called huc8 in the input\_data.gdb)

\item {} 
Switch selection
\begin{itemize}
\item {} 
This will show you streamlines that are outside the LPU

\item {} 
And are probably flowing away

\item {} 
But do a spot check on a few just to make sure

\item {} 
If there are a lot flowing into your LPU from outside, your LPU boundary may need adjusting.

\end{itemize}

\item {} 
Delete the clearly-external streamlines

\end{itemize}

\item {} 
Examine streamlines that cross the boundary
\begin{itemize}
\item {} 
Only one streamline (the outlet) should cross the LPU boundary
\begin{itemize}
\item {} 
Leave this one alone

\item {} 
It should extend multiple pixels’-worth beyond the LPU boundary

\item {} 
If you omit this step, your LPU will fill up like a bathtub during the HydroDEM process and look really trippy

\end{itemize}

\item {} 
All the others need to be dealt with
\begin{itemize}
\item {} 
As you find them consider:
\begin{itemize}
\item {} 
Is this a genuine stream that I need to re-draw my LPU boundary?

\item {} 
Is this a canal (and I can safely delete it)?

\end{itemize}

\item {} 
Trim the initiation part of streamlines crossing the boundary to at least several pixels’-worth from the boundary

\end{itemize}

\item {} 
If you omit this step, water will leak out the edges of your LPU

\end{itemize}

\item {} 
NHD-WBD Intersect Tool: this section is no longer supported by the toolbox

\end{itemize}


\subparagraph{\sphinxstylestrong{TopoGrid}}
\label{\detokenize{ex_1:topogrid}}
If you are going to run TopoGrid, please refer to the instructions in exercise 1b: TopoGrid


\paragraph{\sphinxstyleemphasis{Step 5: Toolbox 4. HydroDEM Tools}}
\label{\detokenize{ex_1:step-5-toolbox-4-hydrodem-tools}}
Kitty’s Note: If you are using data that is not either 30 ft or 10 m data, use the second Bathymetric Tool, called “Bathymetric Gradient Setup for 30m States.” The usual script checks the data for conformance to the 30 ft/10m standard, and will fail on any other conditions. The second version was created by Bob Ourso to use on Alaska (30m) data, but modified to accept anything by Kitty for SC (10ft) data. Bob and I “jailbreaked” the script to accept any size pixels.


\subparagraph{\sphinxstylestrong{A. Bathymetric Gradient Setup from ArcCat}}
\label{\detokenize{ex_1:a-bathymetric-gradient-setup-from-arccat}}\begin{enumerate}
\def\theenumi{\arabic{enumi}}
\def\labelenumi{\theenumi .}
\makeatletter\def\p@enumii{\p@enumi \theenumi .}\makeatother
\item {} 
Run ‘Bathymetric Gradient Setup’ on a local HUC to prepare 2 grids for the HydroDEM program

\item {} 
Set the ‘Output Workspace’ to the local HUC workspace (01091111)

\item {} 
Set the ‘DEM’ grid
\begin{itemize}
\item {} 
Normally you would use ‘dem\_raw’ here.

\item {} 
If you used TopoGrid, use the TopoGrid output dem here (usually called “topogr\_gr”)

\end{itemize}

\item {} 
Set the ‘Dissolved HUC8 Dataset’ to ‘huc8’

\item {} 
Set the NHDArea to the same name in the source NHD feature dataset (under the ‘input\_data’ file geodatabase).

\item {} 
Set the NHDFlowline to ‘NHDFlowline’. (NOTE: Do NOT use NHDFlowline\_orig from the NHD here. You want the version you edited.)

\item {} 
Set the NHDWaterbody to the same name in the source NHD feature dataset (under the ‘input\_data’ file geodatabase).

\item {} 
Set the NHDFlowline Selection Buffer to 5 meters (default)

\item {} 
Ensure the Checkbox for ‘NHD Data already projected’ is checked

\item {} 
Click OK

\end{enumerate}

\sphinxstyleemphasis{Two grids are created: wb\_srcg (waterbody areas) and nhd\_wbg (flowline cells).}

\begin{figure}[htbp]
\centering
\capstart

\noindent\sphinxincludegraphics{{bathgradsetup}.png}
\caption{Figure: Sample inputs for the Bathymetric Gradient Setup tool}\label{\detokenize{ex_1:id14}}\end{figure}

Enter ArcMap and view these grids in context with the source NHD.  Quit out of ArcMap without saving. (NOTE: When you are working on your own data, you may see cases in which the NHDWaterbody or NHDArea may need to be edited in order to prevent imposing a gradient on areas that shouldn’t be done. This would be quite rare, but if you do see this, copy the feature class to an “edit” one like we have done for NHDFlowline and inwall\_edit, so you will know there have been edits. Then use the edited feature classes as inputs to this tool, i.e. in steps 1d and/or 1f above.)

Repeat ‘Bathymetric Gradient Setup’ for 01092222, 01093333, and 01094444


\subparagraph{\sphinxstylestrong{B. Coastal Processing}}
\label{\detokenize{ex_1:b-coastal-processing}}\begin{itemize}
\item {} 
(Exercises Note: These are not coastal hucs, so we are skipping the Coastal Processing step. Otherwise we will need to do coastal processing.  )

\item {} 
Create a polygon feature class that covers the entire footprint of your local folder study area (huc-8 or equivalent)
\begin{itemize}
\item {} 
Call it “LandSea”

\item {} 
Add an integer field called Land

\item {} 
Divide your foot print by categories of land/sea/other
\begin{itemize}
\item {} 
Sources for this could be NHDWaterbodies, Coastline, local lidar shoreline features

\end{itemize}

\item {} 
There will be three categories of features in your feature class
\begin{itemize}
\item {} 
Sea or ocean
\begin{itemize}
\item {} 
Value = -1

\item {} 
This will be lowered to the Sea Level specified in the tool field

\end{itemize}

\item {} 
Areas on land that are truly below sea level (such as a quarry, Death Valley, etc)
\begin{itemize}
\item {} 
Value = 0

\item {} 
This will not be changed by the tool

\end{itemize}

\item {} 
Land areas that are above sea level
\begin{itemize}
\item {} 
Value = 1

\item {} 
Any cells in these areas with elevation of 0 or lower will be raised to 1

\end{itemize}

\end{itemize}

\end{itemize}

\item {} 
The tool will create a dem called “dem\_sea”
\begin{itemize}
\item {} 
This will be used as the input dem for PreHydroDEM

\item {} 
If you used Topogrid, input TopoGrid

\item {} 
If you didn’t use Topogrid, use dem\_raw

\end{itemize}

\end{itemize}


\subparagraph{\sphinxstylestrong{C. Pre HydroDEM Processing from ArcCat}}
\label{\detokenize{ex_1:c-pre-hydrodem-processing-from-arccat}}
Run the ‘Pre HydroDEM Processing’ tool to create the input coverages for HydroDEM (from the source file geodatabases), and define the input DEM as either the raw DEM (‘dem\_raw’).  The chosen DEM will be copied and named ‘dem’
\begin{enumerate}
\def\theenumi{\arabic{enumi}}
\def\labelenumi{\theenumi .}
\makeatletter\def\p@enumii{\p@enumi \theenumi .}\makeatother
\item {} 
Output Workspace: select the ‘01091111’ workspace

\item {} 
Dissolved HUC8 Dataset: ‘huc8’ (in the ‘input\_data’ gdb

\item {} 
Dendritic NHD Dataset: ‘NHDFlowline’

\item {} 
Inner Wall Dataset: ‘inwall\_edit’

\item {} 
DEM to be used in HydroDEM:
\begin{itemize}
\item {} 
If you did Coastal Processing, use dem\_sea

\item {} 
If you skipped Coastal Processing:
\begin{itemize}
\item {} 
Use ‘dem\_raw’ if you didn’t use TopoGrid

\item {} 
Use topogr\_gr if you ran TopoGrid

\end{itemize}

\end{itemize}

\item {} 
Sink Points Dataset (optional): leave blank

\item {} 
Run the program (‘OK’)

\end{enumerate}

Run the remaining 3 HUCs through ‘Pre HydroDEM Processing’ (the ‘Sink Points Dataset (optional)’ menu item will be left blank for all three).

\sphinxstyleemphasis{Grids Created: dem}

\sphinxstyleemphasis{Coverages Created: huc8, inwall, nhdrch}

\begin{figure}[htbp]
\centering
\capstart

\noindent\sphinxincludegraphics{{prehydrodem}.png}
\caption{Figure: Sample inputs for the Pre-HydroDEM Processing tool}\label{\detokenize{ex_1:id15}}\end{figure}


\subparagraph{\sphinxstylestrong{D. Run HydroDEM.aml from ArcMap}}
\label{\detokenize{ex_1:d-run-hydrodem-aml-from-arcmap}}
Make sure the directory is refreshed and not being accessed by any ArcMap or Windows Explorer window.  It may make sense to close ArcCatalog before you run step D and open it later after HydroDem is over if you need it again.

Run the ‘HydroDEM.aml’ tool on 01091111 (the remaining 3 will be run in batch mode, but here we get to showcase the “validation code” for Python scripts, which populates fields \textgreater{} a thanks to Curtis Price for this code. You can see the code on the Validation tab if you right-click on the script in ArcToolbox and go to Properties.)

Output Workspace: select the ‘01091111’ workspace

The remaining fields (including the optional ‘Drain plug coverage’ field) are populated!

NOTE:
\begin{itemize}
\item {} 
if you used the Coastal adjustments, change the DEM to be Used in HydroDem field to the dem output from the Coastal adjustment.

\item {} 
If you used TopoGrid, you are okay, because the PreHydroDEM should have copied your topogr\_gr raster to be the “dem” raster that RunHydroDEM is looking for

\item {} 
copy the snapgrid into the huc folder, and set that as the Snapgrid, instead of the one that is auto populated

\item {} 
Click OK

\item {} 
View results in ArcMap.  Drape the NHDFlowline feature class and drain\_plugs polygon layer on top of the fac grid.  Scan around.

\item {} 
examine the dem\_enforced and zoom back to the nhd\_inwall\_intersect areas that were edited.  make sure the inwalls only have one outlet for each stream.

\item {} 
Quit out of ArcMap without saving

\item {} 
Run the ‘HydroDEM.aml’ tool in batch mode for the other 3 HUCs

\item {} 
Right click the tool and select ‘Batch’

\item {} 
In the empty ‘Output Workspace’ field, right click and select ‘Browse’

\item {} 
Hilite all the remaining 3 HUC folders and select ‘Add’

\item {} 
Back in the batch window, with the 3 newly populated rows (for ‘Output Workspace’) selected, click the check values check box on the right.  All other fields are populated!

\item {} 
Click OK to run the batch process

\item {} 
View some results in ArcMap.  Quit out of ArcMap without saving

\end{itemize}

Grids Created in HUC folder: buffg, dem\_enforced, dem\_ridge8, dem\_ridge8wbd, elevgrid, eucd, fac, fdr, fil, inwallg, inwallg\_tmp, nhdgrd, ridge\_exp, ridge\_nl, ridge\_w

\begin{figure}[htbp]
\centering
\capstart

\noindent\sphinxincludegraphics{{runhydrodem}.png}
\caption{Figure: Sample inputs for the RunHydroDEM.aml tool}\label{\detokenize{ex_1:id16}}\end{figure}

NOTE: the results will contain some very strange looking stuff, this is normal.

\begin{figure}[htbp]
\centering
\capstart

\noindent\sphinxincludegraphics{{hydrodeminprogress}.png}
\caption{Figure: Sample outputs of the HydroDEM process}\label{\detokenize{ex_1:id17}}\end{figure}


\subparagraph{\sphinxstylestrong{E. Flow Accumulation Adjust (Need this when joining globals)}}
\label{\detokenize{ex_1:e-flow-accumulation-adjust-need-this-when-joining-globals}}

\subparagraph{\sphinxstyleemphasis{Overview}}
\label{\detokenize{ex_1:id1}}
Flow Accumulation needs to be adjusted at the inlets of downstream HUCs to account for the flow coming into the HUC (which needs to be recognized for the ArcHydro tools to work correctly). By default, local DEM derivatives like flow accumulation stand alone and do not recognize incoming flow. This tool adjusts for incoming flow.


\subparagraph{\sphinxstyleemphasis{Regular Flow Accum Adjust}}
\label{\detokenize{ex_1:regular-flow-accum-adjust}}
\sphinxstyleemphasis{Workflow}
\begin{itemize}
\item {} 
Open the ‘Flow Accum Adjust’ tool

\item {} 
Downstream (receiving FAC):  Add in the FAC (flow accumulation) grid under 01094444 (which is the only receiving HUC in this exercise)

\item {} 
Upstream FAC(s):  Add in each of the three upstream FAC grids by surfing into each local workspace

\item {} 
Click OK

\item {} 
Open ArcMap and view the results.
\begin{itemize}
\item {} 
Load the new ‘fac\_global’ for 01094444

\item {} 
Compare to the original ‘fac’

\item {} 
There should be a lighter area where the upstream flow enters now

\end{itemize}

\end{itemize}

\sphinxstyleemphasis{Troubleshooting}
\begin{itemize}
\item {} 
Sometimes this process will fail when connecting multiple upstream facs to one downstream receiving fac, or the values being connected are very large. If this happens, update the downstream/receiving fac iteratively, such as
\begin{itemize}
\item {} 
Fac 1 \(\rightarrow\) Fac 4

\item {} 
Fac 2 \(\rightarrow\) Fac 4b

\item {} 
Fac 3 \(\rightarrow\) Fac 4c

\item {} 
etc

\end{itemize}

\item {} 
If this still doesn’t work, you can use the simple flow accum adjust tool

\end{itemize}


\subparagraph{\sphinxstyleemphasis{Simple Flow Accum Adjust}}
\label{\detokenize{ex_1:simple-flow-accum-adjust}}\begin{itemize}
\item {} 
If you are using the simple adjust, make sure the dot is on the first raster cell where things come in

\end{itemize}

\sphinxstylestrong{F. b) Post HydroDEM Processing}

See Exercise 2 for these directions


\subparagraph{\sphinxstyleemphasis{StreamStats Recommended Advanced Settings and Thresholds}}
\label{\detokenize{ex_1:streamstats-recommended-advanced-settings-and-thresholds}}
\sphinxstylestrong{Note:} the commas in the numbers above are for clarity. Enter the numbers in the forms without commas


\begin{savenotes}\sphinxattablestart
\centering
\begin{tabulary}{\linewidth}[t]{|T|T|T|T|T|T|T|}
\hline
\sphinxstyletheadfamily 
Post
HydroDEM
Processing
Thresholds
(Suggested
values
&\sphinxstyletheadfamily 
10-m
DEMs
&\sphinxstyletheadfamily 
30-m
DEMs
&\sphinxstyletheadfamily 
30-ft
DEMs
&\sphinxstyletheadfamily 
10-ft
DEMs
&\sphinxstyletheadfamily 
Exercise
datasets
&\sphinxstyletheadfamily 
3-m
DEMs
\\
\hline
Threshold 1
(Stream
initial
threshold in
cells)
&
150,000
&
50,000
&
150,000
&
450,000
&
8000
&
1,666,667
\\
\hline
Threshold 2
(Detailed
stream grid
threshold in
cells)
&
900
&
100
&
900
&
9,688
&
900
&
10,000
\\
\hline
\end{tabulary}
\par
\sphinxattableend\end{savenotes}


\subsection{Exercise 1b: TopoGrid}
\label{\detokenize{ex_1b:exercise-1b-topogrid}}\label{\detokenize{ex_1b::doc}}
Original StreamStats tool development by Pete Steeves, Al Rea, and Martyn Smith

Transcribed by Kitty Kolb and Pete Steeves, August 2019


\subsubsection{Introduction}
\label{\detokenize{ex_1b:introduction}}
TopoGrid was originally created in Australia as AnuDEM. The algorithms were ingested into ESRI during the 1990s. Originally used in a command line ArcINFO environment, a Python wrapper was created in 2009. More documentation can be found at \sphinxurl{https://www.usgs.gov/core-science-systems/ngp/national-hydrography/nhd-watershed-tool}

TopoGrid enforces drainages, although it does not burn them in like HydroDEM. Enforcing is what can be referred to as a soft break-line process. It sees the stream line and says ‘I’m going to massage the elevation data, to work with that location.” It gets things close, but not precise. It works with the subsequent burning, because then the hard break (the burning/trenching) does not need to be as wide in search of the raster stream. And so, you get to keep much more of the local topology around the stream. If you just do burning alone, and use a small swath tolerance, sometimes the raster stream is not found and you get parallel delineation effects. In general, Topogrid is recommended to improve your DEM in relation to your hydrography. This is particularly helpful in flat terrain where the synthetic streams (derived from a DEM) tend to be relatively inaccurate in relation to the mapped streams.  Caution should be taken to mix and match resolutions, sources and scales.


\paragraph{\sphinxstyleemphasis{Setup}}
\label{\detokenize{ex_1b:setup}}\begin{itemize}
\item {} 
Open a session of either ArcMap or ArcCatalog (either one will work)

\item {} 
Click to run the TopoGrid script in the StreamStats Toolbox.

\item {} 
Point the tool to your workspace

\end{itemize}

\begin{figure}[htbp]
\centering
\capstart

\noindent\sphinxincludegraphics{{topogridinputs}.png}
\caption{Figure: Sample inputs for the Topo Grid tool}\label{\detokenize{ex_1b:id1}}\end{figure}
\begin{itemize}
\item {} \begin{description}
\item[{Open the Environments menu}] \leavevmode\begin{itemize}
\item {} 
Navigate to Processing Extent

\item {} 
Set the Snapping Grid to be the “dem\_raw” file

\item {} 
Click “Ok” to return to the main input window

\end{itemize}

\end{description}

\end{itemize}

NOTE: after you process your first local folder, always navigate back to the first dem\_raw file that you processed, so that the study area is consistent.

\begin{figure}[htbp]
\centering
\capstart

\noindent\sphinxincludegraphics{{topogridenvsettings}.png}
\caption{Figure: Example of setting the snap raster in TopoGrid environment settings}\label{\detokenize{ex_1b:id2}}\end{figure}
\begin{itemize}
\item {} 
Click OK to run TopoGrid

\item {} 
It will run for a long time, depending on your VIP settings and the GRIDALLOC values

\end{itemize}


\subsection{Exercise 2: Setting up the Local and Global Geodatabases}
\label{\detokenize{ex_2:exercise-2-setting-up-the-local-and-global-geodatabases}}\label{\detokenize{ex_2::doc}}

\subsubsection{Document overview}
\label{\detokenize{ex_2:document-overview}}
(This exercise is run on data in the ‘global\_exercise’ workspace)

Written by Al Rea and Pete Steeves, May 2010

Updated by Kitty Kolb and Bob Ourso, July 2015 to reflect ArcGIS 10x; additional revisions by Kitty Kolb and Pete Steeves, March \& July 2019


\paragraph{\sphinxstyleemphasis{Process Local HUC ArcHydro Geodatabases}}
\label{\detokenize{ex_2:process-local-huc-archydro-geodatabases}}
The next step, which we call “Post HydroDEM Processing” goes through several steps from the Terrain Preprocessing menu of the ArcHydro Tools and the ArcHydro Python Toolbox. There are two different models in the StreamStats toolbox, one for use when you have sinks or coastlines (which are treated like sinks), and the other for when you have no coastlines or sinks in your huc. Both are provided as PDFs and handouts. Review them to see what is happening.

Note: portions of this step have been deprecated, since the PostHydroDEM tools no longer work in ArcGIS versions later than 10.3.  The directions assume you are manually stepping through the toolbar instead of using the StreamStats Toolbox


\subparagraph{\sphinxstylestrong{Introduction}}
\label{\detokenize{ex_2:introduction}}\begin{itemize}
\item {} 
StreamStats needs the hucpoly when you delineate to look for the local data for delineations (it uses the “NAME” attribute in the global hucpoly to do this)

\item {} 
The hucpoly is like an index for the local folders

\item {} 
Even if you have only frontal hucs and no global hucs, you still need to create the global geodatabase, or else it won’t know where to look for the data.

\item {} 
The three layers have a relationship to each other- hucpoly, streams, huc\_net\_junctions
\begin{itemize}
\item {} 
99\% of the time in StreamStats, you’ll be doing local delineations, but other times there are large basins that encompass multiple hucs

\item {} 
When you click on a point, it asks, “am I on the global streams layer” If yes, then knows to include upstream basins by tracing upstream on the streams layer.

\item {} 
That relationship grabs the upstream areas and knows to include them; sets them aside. Delineates local piece of basin, then glues it to the upstream larger chunks, dissolves internal boundaries.

\item {} 
If your study area doesn’t have upstream/downstream hucs, will be vestigial; will see things aren’t on the streams layer, but still have to have it for program to work

\end{itemize}

\end{itemize}


\subparagraph{\sphinxstylestrong{StreamStats Recommended Advanced Settings and Thresholds}}
\label{\detokenize{ex_2:streamstats-recommended-advanced-settings-and-thresholds}}
\sphinxstylestrong{Note:} the commas in the numbers above are for clarity. Enter the numbers in the forms without commas


\begin{savenotes}\sphinxattablestart
\centering
\begin{tabulary}{\linewidth}[t]{|T|T|T|T|T|T|T|}
\hline
\sphinxstyletheadfamily 
Post
HydroDEM
Processing
Thresholds
(Suggested
values
&\sphinxstyletheadfamily 
10-m
DEMs
&\sphinxstyletheadfamily 
30-m
DEMs
&\sphinxstyletheadfamily 
30-ft
DEMs
&\sphinxstyletheadfamily 
10-ft
DEMs
&\sphinxstyletheadfamily 
Exercise
datasets
&\sphinxstyletheadfamily 
3-m
DEMs
\\
\hline
Threshold 1
(Stream
initial
threshold in
cells)
&
150,000
&
50,000
&
150,000
&
450,000
&
8000
&
1,666,667
\\
\hline
Threshold 2
(Detailed
stream grid
threshold in
cells)
&
900
&
100
&
900
&
9,688
&
900
&
10,000
\\
\hline
\end{tabulary}
\par
\sphinxattableend\end{savenotes}


\subparagraph{\sphinxstylestrong{Post-HydroDEM Processing}}
\label{\detokenize{ex_2:post-hydrodem-processing}}

\subparagraph{\sphinxstyleemphasis{Post-HydroDEM No Sinks}}
\label{\detokenize{ex_2:post-hydrodem-no-sinks}}

\subparagraph{\sphinxstyleemphasis{Kara’s Method (Probably the one you should use)}}
\label{\detokenize{ex_2:kara-s-method-probably-the-one-you-should-use}}\begin{itemize}
\item {} 
This method was developed by Kara Watson in NJWSC

\item {} 
You will need to add the ArcHydro Tools Toolbar to your display if you haven’t already

\end{itemize}

\begin{figure}[htbp]
\centering
\capstart

\noindent\sphinxincludegraphics{{terrainprepromini}.png}
\caption{Figure: Screen capture of the ArcHydro Tools Toolbar}\label{\detokenize{ex_2:id2}}\end{figure}
\begin{itemize}
\item {} 
This is the equivalent of PostHydroDem tool

\item {} 
The things that you need:
\begin{itemize}
\item {} 
fac or fac\_global

\item {} 
fdr

\item {} 
Load them into ArcMap map document; this sets the projection of your map document

\end{itemize}

\item {} 
Save this in your local folder, give it the same name as your local folder {[}local folder name{]}.mxd
\begin{itemize}
\item {} 
This creates a geodatabase in your local folder with the local folder name

\item {} 
Make sure that your target locations are set for the local folders:

\end{itemize}

\end{itemize}

\begin{figure}[htbp]
\centering
\capstart

\noindent\sphinxincludegraphics{{targetlocations}.png}
\caption{Figure: Screen capture of the Set Target Locations menu on the ArcHydro toolbar}\label{\detokenize{ex_2:id3}}\end{figure}
\begin{itemize}
\item {} 
Go to the Terrain Processing toolbar in ArcHydro

\end{itemize}

\begin{figure}[htbp]
\centering
\capstart

\noindent\sphinxincludegraphics{{streamdefinitiontoolbar}.png}
\caption{Figure: Screen capture of the location of the Stream Definition tool on the ArcHydro Tools Toolbar}\label{\detokenize{ex_2:id4}}\end{figure}
\begin{itemize}
\item {} 
Start with the stream definition step
\begin{itemize}
\item {} 
This step makes the str grid

\item {} 
run this twice

\item {} 
Thresholds
\begin{itemize}
\item {} 
Threshold 1  —\textgreater{} str grid

\item {} 
Threshold 2 —-\textgreater{} str900 (or equivalent) NOTE: for a table of threshold values, see the beginning of this document

\item {} 
Yours will look something like this, although the area (square km) value may be slightly different

\end{itemize}

\end{itemize}

\end{itemize}

\begin{figure}[htbp]
\centering
\capstart

\noindent\sphinxincludegraphics{{str900thresholds}.jpg}
\caption{Figure: Screen capture of the sample inputs for the Stream Definition tool}\label{\detokenize{ex_2:id5}}\end{figure}

\sphinxstylestrong{Note:} Run the following steps on just the str grid:
\begin{itemize}
\item {} 
Stream Segmentation
\begin{itemize}
\item {} 
This creates the stream link grid

\item {} 
leave sink stuff blank unless you have sinks (if you have sinks you should use the Post-Processing with Sinks workflow further down)

\end{itemize}

\end{itemize}
\begin{itemize}
\item {} 
Create the catchment grid
\begin{itemize}
\item {} 
This step uses link grid strlnk from previous step

\item {} 
This can be run from either the Toolbar or the ArcToolbox stages

\end{itemize}

\end{itemize}
\begin{itemize}
\item {} 
Catchment Polygon processing
\begin{itemize}
\item {} 
This can be run from the toolbar

\end{itemize}

\end{itemize}
\begin{itemize}
\item {} 
Drainage line processing
\begin{itemize}
\item {} 
Save your map document (the program sometimes fails at this stage)

\item {} 
This can be run from either the toolbar or the AH Python Tools toolbox

\end{itemize}

\item {} 
Adjoint Catchment processing
\begin{itemize}
\item {} 
Save your map document again (the program sometimes fails at this stage)

\item {} 
Navigate to ArcToolbox \(\rightarrow\) AH Tools Python \(\rightarrow\) Terrain Preprocessing \(\rightarrow\) Adjoint Catchment processing

\item {} 
Populate the fields from your layers, and take the defaults for the rest. Your option for Input Drainage Line Split Table should be blank.

\item {} 
Note: It might tell you at the end that there are loops. I have no idea what this means.

\item {} \begin{description}
\item[{IF THIS STEP FAILS:}] \leavevmode\begin{itemize}
\item {} 
It may give you a yellow triangle and say “general function failure” with an empty feature class for the adjoint catchments

\item {} 
Close your ArcMap session without saving

\item {} 
Re-enter your map document

\item {} 
Delete the empty AdjointCatchment feature class

\item {} 
Add a field to the Drainage Line feature class called ‘DrainID” (long integer)

\item {} 
Run the Adjoint Catchment Processing tool from the toolbar, rather than the toolbox

\item {} 
It should work this time

\end{itemize}

\end{description}

\end{itemize}

\end{itemize}
\begin{itemize}
\item {} 
grids are stored in a folder called Layers
\begin{itemize}
\item {} 
You can move these up one level at the end of processing to be tidy, but Pete S says this is not necessary. It bugs Kitty, so she always does it.

\item {} 
str

\item {} 
str900

\item {} 
cat

\item {} 
strlnk

\end{itemize}

\item {} 
Drainage Point Processing
\begin{itemize}
\item {} 
This step may be vestigial, but it can’t hurt to do it. It is short.

\item {} 
Longest Flow Path Processing occasionally asks for it, depending on your version

\item {} 
Choose Drainage Point Processing from either the AH Toolbar or the AH Python Toolbox

\item {} 
Populate the fields as follows:

\end{itemize}

\end{itemize}


\subparagraph{\sphinxstyleemphasis{Old Directions That No Longer Work}}
\label{\detokenize{ex_2:old-directions-that-no-longer-work}}\begin{enumerate}
\def\theenumi{\arabic{enumi}}
\def\labelenumi{\theenumi .}
\makeatletter\def\p@enumii{\p@enumi \theenumi .}\makeatother
\item {} 
Start a fresh ArcMap document. (Important: Don’t just hit the “New Map Document” button. This will copy the XML setup from the currently open MXD. We want a completely fresh MXD, so you must start ArcMap from scratch. (Don’t save it yet, either. Wait till step 3 below.)

\item {} 
Add the fdr and fac grids from the 01092222 workspace to the mxd.  Be sure to add the grids first, before anything else. This ensures the projection is set right.

\item {} 
Save the MXD as 01092222.mxd in the 01092222 folder. Notice in the folder that extra files appear. There will appear a 01092222.gdb folder, which is a file Geodatabase that will contain the vector layers. Also created is a 01092222.AHD file, which stores the ArcHydro Tools configuration, in XML format, plus a 01092222.xml file. (If you are having trouble in a folder, try deleting these files, including the MXD, and starting over.)

\item {} 
Run the “F. b) Post HydroDEM.aml Processing (No Sinks)” model/tool. Be sure that you are using the appropriate grid layers for each input using the little pulldown arrow. Do NOT use the file folder browse tool to specify the grids. Set the first Threshold to 8000, the second to 900.

\end{enumerate}


\subparagraph{\sphinxstyleemphasis{Post-HydroDEM (With Sinks)}}
\label{\detokenize{ex_2:post-hydrodem-with-sinks}}\begin{itemize}
\item {} 
To Start*

\end{itemize}


\bigskip\hrule\bigskip

\begin{itemize}
\item {} 
Open a new session of ArcMap
\begin{itemize}
\item {} 
Save your map document with the name of your local folder

\item {} 
Add the following files to your view:
\begin{itemize}
\item {} 
fac (or fac\_global)

\item {} 
fdr

\item {} 
sinklink

\item {} 
sinkpoint\_edit (from input\_data.gdb)

\end{itemize}

\item {} 
Note: ArcHydro will default that the output grids are stored in a folder called Layers
\begin{itemize}
\item {} 
You can move these up one level at the end of processing to be tidy, but Pete S says this is not necessary. It bugs Kitty, so she always does it. str

\item {} 
str900

\item {} 
cat

\item {} 
strlnk

\end{itemize}

\end{itemize}

\end{itemize}


\subparagraph{\sphinxstyleemphasis{Stream and Sink Combination}}
\label{\detokenize{ex_2:stream-and-sink-combination}}\begin{itemize}
\item {} 
Define Streams using the Stream Definition tool on the ArcHydro toolbar

\end{itemize}

\begin{figure}[htbp]
\centering
\capstart

\noindent\sphinxincludegraphics{{streamdefinitiontoolbar}.png}
\caption{Figure: Screen capture of the location of the Stream Definition tool on the ArcHydro Tools Toolbar}\label{\detokenize{ex_2:id6}}\end{figure}
\begin{itemize}
\item {} 
This is done twice for separate thresholds (stream grid versus snap grid)
\begin{itemize}
\item {} 
You can find the thresholds in the table further up this document or at the bottom of Exercise 1

\item {} 
First pass \(\rightarrow\) str grid
\begin{itemize}
\item {} 
Note: if you have a fac\_global, use fac\_global instead of the fac

\end{itemize}

\item {} 
Second pass \(\rightarrow\) str900 grid
\begin{itemize}
\item {} 
Name this “str” plus whatever your threshold is. Your threshold may vary based on the pixel size of your application.(See table at the end of Exercise 1 for common pixel sizes)

\item {} 
Note: if you have a fac\_global, use fac\_global instead of the fac

\end{itemize}

\end{itemize}

\end{itemize}

\begin{figure}[htbp]
\centering
\capstart

\noindent\sphinxincludegraphics{{str900thresholds}.jpg}
\caption{Figure: Screen capture of the sample inputs for the Stream Definition tool}\label{\detokenize{ex_2:id7}}\end{figure}
\begin{itemize}
\item {} 
Create a Sink Point Raster
\begin{itemize}
\item {} 
Add your Sink Link Raster to the view if you haven’t already (quite possibly called “sinklink”)

\item {} 
Export or Create a copy of the Sink Link Raster to your Layers folder, and call it “sinkpntgrid”

\item {} 
Add sinkpntgrid to the map document

\item {} 
\sphinxstylestrong{Note:} I don’t see any difference between sinklink and sinkpntgrid, but the official ArcHydro for Stormwater instructions use sinkpntgrid instead of sinklink. I’ve used “sinklink” in place of “sinkpntgrid” and there do not seem to be any deleterious effects.

\end{itemize}

\item {} 
Delineate Sink Watersheds
\begin{itemize}
\item {} 
In ArcToolbox, navigate to ArcHydro Tools Python \(\rightarrow\) Terrain PreProcessing \(\rightarrow\) Sink Watershed Delineation

\item {} 
Populate the input prompts with the files from your map document:

\item {} 
Take the default for the outputs

\item {} 
It will create two layers:
\begin{itemize}
\item {} 
SinkWshGrid \(\rightarrow\) a raster representation of the watersheds draining to the sinks

\item {} 
Sink Watershed \(\rightarrow\) a vector representation of the watersheds draining to the sinks

\end{itemize}

\end{itemize}

\end{itemize}

\begin{figure}[htbp]
\centering
\capstart

\noindent\sphinxincludegraphics{{sinkwatershedtool}.png}
\caption{Figure: Screen capture of the sample inputs for the Sink Watershed Delineation tool}\label{\detokenize{ex_2:id8}}\end{figure}

\begin{figure}[htbp]
\centering
\capstart

\noindent\sphinxincludegraphics{{images/streamsegmenttool}.png}
\caption{Figure: Screen capture showing the location of the Stream Segmentation tool on the ArcHydro toolbar}\label{\detokenize{ex_2:id9}}\end{figure}
\begin{itemize}
\item {} 
Choose the Stream Segmentation task from the ArcHydro Toolbar
\begin{itemize}
\item {} 
Populate the fields from your map document

\item {} 
Note: use your str grid for the Stream Grid

\item {} 
It will think for a while, then say “stream segmentation successfully completed”

\end{itemize}

\end{itemize}

\begin{figure}[htbp]
\centering
\capstart

\noindent\sphinxincludegraphics{{streamsegmentsink}.png}
\caption{Figure: Screen capture of the sample inputs for the Stream Segmentation tool using sinks}\label{\detokenize{ex_2:id10}}\end{figure}

\begin{figure}[htbp]
\centering
\capstart

\noindent\sphinxincludegraphics{{combinestreamsink}.png}
\caption{Figure: Screen capture showing the location of the Combine Stream Link and Sink Link tool on the ArcHydro toolbar}\label{\detokenize{ex_2:id11}}\end{figure}
\begin{itemize}
\item {} 
Choose the Combine Stream Link and Sink Link tool from the ArcHydro Toolbar
\begin{itemize}
\item {} 
Verify that it has set the proper inputs

\item {} 
You may need to manually point it to the Drainage Line feature class that was just created in the last step

\item {} 
This creates a new grid called “lnk”

\end{itemize}

\end{itemize}

\begin{figure}[htbp]
\centering
\capstart

\noindent\sphinxincludegraphics{{combinestreamsinkinputs}.png}
\caption{Figure: Screen capture showing sample inputs for the Combine Stream Link and Sink Link tool}\label{\detokenize{ex_2:id12}}\end{figure}
\begin{itemize}
\item {} 
Choose the Catchment Grid Delineation step from the ArcHydro Toolbar
\begin{itemize}
\item {} 
The Link grid that it asks for will be the link grid you just made.

\item {} 
Verify that the other inputs are correct (It should look like this):

\end{itemize}

\end{itemize}

\begin{figure}[htbp]
\centering
\capstart

\noindent\sphinxincludegraphics{{catgriddelinsink}.png}
\caption{Figure: Screen capture showing sample inputs for the Catchment Grid Delineation tool when your watershed has sinks}\label{\detokenize{ex_2:id13}}\end{figure}
\begin{itemize}
\item {} 
Convert the Catchment grid to polygons
\begin{itemize}
\item {} 
Navigate to the Catchment Polygon Processing tool in Arc Toolbox \(\rightarrow\) ArcHydro Tools Python \(\rightarrow\) Terrain Preprocessing

\item {} 
Choose the cat grid from the previous step. Verify that the inputs are correct:

\end{itemize}

\end{itemize}

\begin{figure}[htbp]
\centering
\capstart

\noindent\sphinxincludegraphics{{catpoly2}.png}
\caption{Figure: Screen capture showing sample inputs for the Catchment Polygon Processing tool}\label{\detokenize{ex_2:id14}}\end{figure}
\begin{itemize}
\item {} 
Create the adjoint catchments
\begin{itemize}
\item {} 
In ArcToolbox, navigate to the Arc Hydro Tools Python \(\rightarrow\) Terrain Preprocessing \(\rightarrow\) Adjoint Catchment Processing tool

\item {} 
Populate the fields with the layers that you’ve run previously

\item {} 
The output flow split table will be empty, because you made a dendrite net(work)

\item {} 
Run the tool

\item {} 
It might tell you at the end that there are loops. I have no idea what this means.

\end{itemize}

\end{itemize}

\begin{figure}[htbp]
\centering
\capstart

\noindent\sphinxincludegraphics{{adjcatproc2}.png}
\caption{Figure: Screen capture showing sample inputs for the Adjoint Catchment Processing tool}\label{\detokenize{ex_2:id15}}\end{figure}
\begin{itemize}
\item {} 
IF THIS STEP (adjoint catchment) FAILS:
\begin{itemize}
\item {} 
It may give you a yellow triangle and say “general function failure” with an empty feature class for the adjoint catchments

\item {} 
Close your ArcMap session

\item {} 
Re-enter your map document

\item {} 
Delete the empty AdjointCatchment feature class

\item {} 
Add a field to the Drainage Line feature class called ‘DrainID” (long integer)

\item {} 
Run the Adjoint Catchment Processing tool from the toolbar, rather than the toolbox

\item {} 
It should work this time

\end{itemize}

\item {} 
Create Drainage Points
\begin{itemize}
\item {} 
From the ArcToolbox \(\rightarrow\) ArcHydro Tools Python \(\rightarrow\) Terrain PreProcessing toolbox, run the Drainage Point Processing tool

\item {} 
Verify that the inputs are correct.

\item {} 
Note: if you have a fac\_global, use the fac\_global instead of the fac

\end{itemize}

\end{itemize}

\begin{figure}[htbp]
\centering
\capstart

\noindent\sphinxincludegraphics{{drainptproc2}.png}
\caption{Figure: Screen capture showing sample inputs for the Drainage Point Processing tool}\label{\detokenize{ex_2:id16}}\end{figure}


\subparagraph{\sphinxstylestrong{Examine your Data}}
\label{\detokenize{ex_2:examine-your-data}}\begin{enumerate}
\def\theenumi{\arabic{enumi}}
\def\labelenumi{\theenumi .}
\makeatletter\def\p@enumii{\p@enumi \theenumi .}\makeatother
\setcounter{enumi}{4}
\item {} 
Look at the results. Do the streams900cells and DrainageLines match well with the input dendrite\_edit? Look at the sinks\_poly coverage. Are there large areas that were filled and flattened? Could these areas be better represented by adding some lines to the dendrite\_edit, e.g. to break through a dam or some other blockage? Use the “raindrop” (Flow Path Tracing) tool to see how the HydroDEM is routing flow. Once you are satisfied, save the mxd and close ArcMap.

\item {} 
Open ArcCatalog.  Browse into the 0109222 workspace. Right-click on and Compact (NOT Compress) the 0109222.gdb geodatabase.

\item {} 
Notice the software also created a “Layers” folder, and it contains several grids. Use ArcCatalog to copy all the grids up one layer to the 01092222 folder, and then delete the Layers folder.

\item {} 
The other 3 HUCs can be processed in a similar manner. For 01094444, add fac\_global instead of fac to the MXD, and specify that by choosing it from the pulldown list of ArcMap grid layers.

\end{enumerate}


\subparagraph{\sphinxstylestrong{Run the Cleanup Workspace tool}}
\label{\detokenize{ex_2:run-the-cleanup-workspace-tool}}\begin{enumerate}
\def\theenumi{\arabic{enumi}}
\def\labelenumi{\theenumi .}
\makeatletter\def\p@enumii{\p@enumi \theenumi .}\makeatother
\setcounter{enumi}{8}
\item {} 
This will clean out many of the temporary data sets that were created in our processes, but not all.
\begin{itemize}
\item {} 
Do not delete all the temporary datasets until you feel comfortable your fdr and fac, etc, are all right. You may need to re-run hydroDEM later on and need the extra datasets.

\item {} 
In addition, delete the spare coverages and the “nhd\_scg” and “wbd\_scg” grids

\end{itemize}

\end{enumerate}
\begin{itemize}
\item {} 
Create the Global Geodatabase *

\end{itemize}


\bigskip\hrule\bigskip

\begin{enumerate}
\def\theenumi{\arabic{enumi}}
\def\labelenumi{\theenumi .}
\makeatletter\def\p@enumii{\p@enumi \theenumi .}\makeatother
\item {} 
In ArcCatalog, create a new File Geodatabase in the global\_exercise folder and call it global.gdb.
\begin{itemize}
\item {} 
The folder location will be your “archydro” folder if you are running your own data

\end{itemize}

\item {} 
Right click on the new global geodatabase and select the New \textgreater{} Feature Dataset
\begin{itemize}
\item {} 
Give it the name ‘Layers’.

\end{itemize}

= Set the coordinate system
\begin{itemize}
\item {} 
If you are using the exercise data, choose the coordinate system by selecting the “USA Contiguous Albers Equal Area Conic USGS.prj” from the Coordinate Systems\textgreater{}Projected Coordinate Systems\textgreater{}Continental\textgreater{}North America.

\item {} 
If you are working on actual data and not exercise data, use your local projection instead of USGS Albers

\item {} 
NOTE: It is important to choose this .prj file from ESRI’s standard prj’s. DO NOT make your own projection- ALWAYS USE AN ESRI standard one

\end{itemize}
\begin{itemize}
\item {} 
Default the next menu (vertical coordinate system) to none and default the last menu (tolerance settings).

\end{itemize}

\end{enumerate}
\begin{itemize}
\item {} 
Global HucPoly *

\end{itemize}


\bigskip\hrule\bigskip



\subparagraph{** Introduction **}
\label{\detokenize{ex_2:id1}}\begin{itemize}
\item {} 
The global hucpoly is like an index. The StreamStats will check the hucpoly and the streams layers first, even if you have only frontal hucs and no global hucs.

\item {} 
The huc\_net is like straws reconnecting the watersheds that HydroDEM spent all the time walling in. It tells StreamStats which upstream and downstream hucs flow into each other

\end{itemize}


\subparagraph{** Methods of making a Global HucPoly **}
\label{\detokenize{ex_2:methods-of-making-a-global-hucpoly}}
Now we will start creating the layers for the global geodatabase.  We will start with the global hucpoly feature class. There are two ways of doing this:


\subparagraph{\sphinxstyleemphasis{Blob Method (Preferred)}}
\label{\detokenize{ex_2:blob-method-preferred}}\begin{itemize}
\item {} 
Open a new ArcMap.mxd and save as ‘makeglobal.mxd’ in the upper level (archydro) workspace
\begin{itemize}
\item {} 
ArcHydro will automatically create a project gdb called “makeglobal.gdb”

\item {} 
Ignore this: the target for the layers you will be creating should be global.gdb

\end{itemize}

\item {} 
Add the fdr from your first local workspace to act as a snap grid. This has two purposes:
\begin{itemize}
\item {} 
Snapping to consistent origin coordinates is a critical element to building nearly all of your ArcHydro raster datasets.

\item {} 
It sets the projection for your hucpoly

\item {} 
Be VERY sure you have the right projection in your map document.

\end{itemize}

\item {} 
Using Map Algebra, create a new raster using a Con statement:
\begin{itemize}
\item {} 
The statement should read (Con “fdr” ≥0, 130). You can use any number you like that is greater than 128 for the last number in the command.

\item {} 
Name your output raster whatever you would like. We’ll call it “blob” for the purposes of this exercise

\item {} 
Save it to a scratch area

\end{itemize}

\end{itemize}

\begin{figure}[htbp]
\centering
\capstart

\noindent\sphinxincludegraphics{{confdrblob}.png}
\caption{Figure: Screen capture of the map algebra statement for converting the fdr grid to a blob}\label{\detokenize{ex_2:id17}}\end{figure}
\begin{itemize}
\item {} 
The blob raster now has the same footprint, projection, pixel size, and snapping as your fdr grid, but is a solid blob.
\begin{itemize}
\item {} 
Use Conversion Tools \(\rightarrow\) From Raster \(\rightarrow\) Raster to Polygon to create a new polygon feature class representing the footprint of your watershed.

\item {} 
You can name it whatever you like, but for this exercise we’ll call it “blob\_poly.”

\item {} 
Uncheck the “simplify polygons feature class”

\end{itemize}

\end{itemize}

\begin{figure}[htbp]
\centering
\capstart

\noindent\sphinxincludegraphics{{blobrastopoly}.png}
\caption{Figure: Screen capture of sample inputs for the Raster to Polygon tool}\label{\detokenize{ex_2:id18}}\end{figure}
\begin{itemize}
\item {} 
You may find that there are small dangly features hanging from the edge of your main feature in the feature class. Jennifer Sharpe suggests an equal/alternate method to open the attribute table for the feature class, where there should be only 1 record.
\begin{itemize}
\item {} 
If you have more than one feature in the feature class, dissolve these into one so that there is one multipart polygon feature class.

\end{itemize}

\item {} 
Make a “blob\_poly” for every local fdr in your study area. Give each one a unique identifier, like “blob\_polyxxxxx.”

\item {} 
Remove the fdr and blob grids from your map document, leaving the “blob\_poly” feature classes

\item {} 
Merge the “blob\_poly” feature classes into a single feature class within your global geodatabase called “hucpoly”.
\begin{itemize}
\item {} 
Insert a text field called “Name,” and populate it with the name of each local watershed

\item {} 
Delete the “ID” and “gridcode” fields

\end{itemize}

\item {} 
Check for overlaps and gaps in your hucpoly between the local footprints
\begin{itemize}
\item {} 
Determine your tolerance for imperfections. If you need help performing triage on gaps, you can consult the document “If You Have Gaps in Your Data”

\item {} 
If you have small gaps of a single pixel on a ridgeline, this will probably not break StreamStats;
\begin{itemize}
\item {} 
You may safely ignore them

\item {} 
Unless you are a perfectionist like Kitty who breaks out into hives thinking about gaps

\end{itemize}

\item {} 
If you have gaps of more than two or three pixels on a ridgeline, you should probably fix these.
\begin{itemize}
\item {} 
Large gaps could potentially make holes in global watersheds and users will contact the support team about it.

\end{itemize}

\item {} 
If you have any overlaps these will need to be fixed unless your study area is entirely frontal hucs (every huc is an outlet huc and nothing is upstream of a huc).
\begin{itemize}
\item {} 
Overlaps will break upstream traces, channel slope, and global watersheds, as well as inflating drainage area calculations by the amount of the overlap.

\end{itemize}

\end{itemize}

\item {} 
If you can’t decide what to do, please see the training document “If You Have Gaps in Your Hucpoly” for more examples.

\item {} 
Solving gaps and overlaps:
\begin{itemize}
\item {} 
Redraw your huc\_8/wbd boundary to include or exclude the offending pixels

\item {} 
Re-run the StreamStats steps for the affected local folders
\begin{itemize}
\item {} 
Depending on how severe the affected area is

\item {} 
You may have to rerun the steps from the beginning (1a) for only that local folder

\item {} 
You can leave the other local folders as they are if they are not affected; they do not have to go through the steps again

\item {} 
You will need a new huc\_8 boundary, a new huc\_8 buffer, a new dem, a new fdr, fac, etc. Your flowlines should not have radically changed.

\end{itemize}

\end{itemize}

\item {} 
You can delete blob and blob\_poly when you are done with this step

\end{itemize}


\subparagraph{\sphinxstyleemphasis{Huc8Index Method (Old Way)}}
\label{\detokenize{ex_2:huc8index-method-old-way}}\begin{itemize}
\item {} 
Open a new ArcMap.mxd and save as ‘makeglobal.mxd’ in the upper level workspace.  Add the ‘huc8index’ feature class, which can be found in the input\_data geodatabase. This is the projected version of the WBD hucs. Double check to make sure it is the correct projection.

\item {} 
Add a dem or fdr from any of the local workspaces to act as a snap grid.  Snapping to consistent origin coordinates is a critical element to building nearly all of your ArcHydro raster datasets.

\item {} 
In ArcToolbox, find the Conversion Tools toolbox. Under “To Raster”, start the Polygon to Raster tool.
\begin{itemize}
\item {} 
Choose huc8index from the layer pulldown for the Input Features.

\item {} 
Choose HUC\_8 for the Value field.

\item {} 
For the output raster, this is a temporary raster, so you can let it default. (Later you will want to kill the raster.)

\item {} 
Let the other parameters default, except set the Cell Size to 10.

\item {} 
IMPORTANT: Before you hit OK, hit the Environments button. Expand the General Settings. Set the Extent to “Same as layer huc8index”. Select the dem layer as the Snap Raster. (If you don’t do this, the output grid cells will not align properly.)

\item {} 
Open Properties of the output raster, and verify that the extent coordinates are even multiples of 15, like all our other grids.

\end{itemize}

\item {} 
In ArcToolbox, find the Conversion Tools toolbox. Under “From Raster” start the “Raster to Polygon” tool.
\begin{itemize}
\item {} 
Choose the raster you made above as the Input raster.

\item {} 
Choose HUC\_8 as the field.

\item {} 
Set the output polygons to go into the global.mdb geodatabase we created earlier, under the “Layers” feature dataset, and named “hucpoly”.

\item {} 
Uncheck the Simplify polygons checkbox. Click OK.

\item {} 
Symbolize the hucpoly features with a hollow symbol and some colored outline. Zoom in and compare it to the original huc8index, and to some of the grids in the local workspace. It should align perfectly with the grids, but will be a jagged, raster-like representation of the WBD.

\item {} 
NOTE: you should not have holes or overlaps in your hucpoly.
\begin{itemize}
\item {} 
Run a topology to find overlaps and gaps between the features in your hucpoly feature class

\item {} 
If you have large holes, you have problems and need to fill them with data. Adjust your original wbd input shapefile and start from the beginning of grid processing for that local folder

\item {} 
Small holes along ridgelines, etc, should be fixed in an ideal world. If you are pressed for time, you can decide whether it is a good return of your time to start over for a small area. How OCD are the users in your state? Some states have more vocal users than others.

\item {} 
All overlaps need to be fixed UNLESS you have a state where there are only local/outlet hucs and not global hucs, in which case you could let it slide, as long as they aren’t egregiously overlapping.  If you have upstream/downstream hucs that overlaps, this will cause massive problems if you want to do upstream tracing or longest flow path.

\end{itemize}

\end{itemize}

\end{itemize}


\paragraph{\sphinxstyleemphasis{Global Streams}}
\label{\detokenize{ex_2:global-streams}}
Now let’s create the global ‘streams’ feature class. The purpose of this is like a signpost for the program to know where a delineation point is in the study area, and whether it needs to include upstream information. You have to have at least one stream in the streams feature class, even if you have an entirely frontal study area of all outlet local folders


\subparagraph{\sphinxstylestrong{Outlet Trace}}
\label{\detokenize{ex_2:outlet-trace}}\begin{itemize}
\item {} 
Add the 01091111/fdr and 01091111/str grids.  Turn off the fdr grid.

\item {} 
Zoom into the outlet area of 01091111.

\end{itemize}

\begin{figure}[htbp]
\centering
\capstart

\noindent\sphinxincludegraphics{{outlet}.png}
\caption{Figure: Illustration of the outlet of a local processing unit along with its stream grid}\label{\detokenize{ex_2:id19}}\end{figure}
\begin{itemize}
\item {} 
Select the flowpath tracing tool from the ArcHydro toolbar

\end{itemize}

\begin{figure}[htbp]
\centering
\capstart

\noindent\sphinxincludegraphics{{flowpathbutton}.png}
\caption{Figure: Illustration of the location of the Flow Path Tracing tool on the ArcHydro toolbar}\label{\detokenize{ex_2:id20}}\end{figure}
\begin{itemize}
\item {} \begin{description}
\item[{NOTE: the order of the following steps may be reversed, depending on which version of ArcGIS you are using}] \leavevmode\begin{itemize}
\item {} 
In the menu that pops up, specify the 01091111/fdr grid

\item {} 
Click on a cell anywhere upstream along the 01091111/str grid (I typically select a cell at least 20 cells upstream)

\item {} 
Click OK if needed

\end{itemize}

\end{description}

\end{itemize}

\begin{figure}[htbp]
\centering
\capstart

\noindent\sphinxincludegraphics{{flowtraceinput}.png}
\caption{Figure: Screen capture of the input for the Flow Path Tracing tool}\label{\detokenize{ex_2:id21}}\end{figure}
\begin{itemize}
\item {} 
The tool will trace your fdr grid to the outlet of your local fdr with a graphic

\end{itemize}

\begin{figure}[htbp]
\centering
\capstart

\noindent\sphinxincludegraphics{{flowtracegraphic}.png}
\caption{Figure: Screen capture showing the graphic result of the Flow Path Tracing tool}\label{\detokenize{ex_2:id22}}\end{figure}


\subparagraph{\sphinxstylestrong{Convert to Features (Single Streams)}}
\label{\detokenize{ex_2:convert-to-features-single-streams}}\begin{itemize}
\item {} 
Load the XTools Toolbar, if it isn’t already.

\item {} 
Use the pointer on the Drawing toolbar to select your graphic (it will have drag handles around the selection)

\end{itemize}

\begin{figure}[htbp]
\centering
\capstart

\noindent\sphinxincludegraphics{{graphicselected}.png}
\caption{Figure: Screen capture showing the selected Flow Path Trace graphic with the drag handles}\label{\detokenize{ex_2:id23}}\end{figure}
\begin{itemize}
\item {} 
In Xtools Pro, go to Feature Conversions \textgreater{} Convert Graphics to Shapes
\begin{itemize}
\item {} 
Check the default box to choose the input graphics layer.

\end{itemize}

\end{itemize}

\begin{figure}[htbp]
\centering
\capstart

\noindent\sphinxincludegraphics{{xtoolsgraphicsshape}.png}
\caption{Figure: Screen capture of the sample inputs for the Convert Graphics to Shape tool in XTools}\label{\detokenize{ex_2:id24}}\end{figure}
\begin{itemize}
\item {} 
Switch graphic type to polyline

\item {} 
Output Feature Class \(\rightarrow\)  Save as Type “Feature Class”

\item {} 
Surf down to the global layers feature dataset.  Name the feature class ‘strm\_1111’.  (If you are working on your own data, it should be something indicative of your local folder instead of the 1111) Click save.

\item {} 
Once back in the main menu, uncheck ‘Add ID Field’ and ‘Add Text field’

\item {} 
It is up to you whether you are feeling lucky/confident whether you check “Delete graphics after conversion.” .

\item {} 
Click Run.

\item {} 
Remove the str and fdr grids from the legend

\item {} 
Select the graphics (a line and a dot) and delete

\item {} 
Repeat the above steps for 2222 and 3333. Name the outputs ‘strm\_2222’ and ‘strm\_3333’.
\begin{itemize}
\item {} 
Note: Most States will have some border/outlet/frontal hucs that flow out of the State and have no upstream hucs flowing into them.

\item {} 
These won’t connect upstream and downstream
\begin{itemize}
\item {} 
You do not have to include these

\item {} 
But it never hurts to include them

\item {} 
For these disconnected hucs, you will have little tiny stubs at the exit of each huc

\end{itemize}

\item {} 
If your state is all frontal/outlet hucs, you need to have at least one stub to make the geometric network
\begin{itemize}
\item {} 
StreamStats will still look for the networked streams feature class when it is delineating.

\item {} 
So even in a study area that is entirely frontal hucs with nothing upstream, you’ll need at least one stub to make a geometric network about.

\end{itemize}

\end{itemize}

\end{itemize}


\subparagraph{\sphinxstylestrong{Convert to Features (Multiple Streams)}}
\label{\detokenize{ex_2:convert-to-features-multiple-streams}}\begin{itemize}
\item {} 
For 4444, repeat the above steps, but run the process at the inlet cell for each of the 3 inlets (1111, 2222, and 3333).  Run all 3 Flow Path Tracing graphics before converting to polylines.  Name the output ‘strm\_4444’
\begin{itemize}
\item {} 
Start Editing with the global geodatabase as your target

\item {} 
Set selectable layers to ‘strm\_4444’.

\item {} 
Select all 3 lines in Strm\_4444 with the edit tool

\item {} 
Load the Topology Toolbar (if not already loaded) into your ArcMap session.  On the Topology toolbar, click the “Planarize Lines” tool to remove all the overlaps and to create nodes at the junctions. Accept the default cluster tolerance and Click ok.  Strm\_4444  should now have 5 line features.

\end{itemize}

\end{itemize}


\subparagraph{\sphinxstylestrong{Merging Local Streams into Global}}
\label{\detokenize{ex_2:merging-local-streams-into-global}}\begin{itemize}
\item {} 
Use the ArcToolBox Merge tool (under Data Management Tools \(\rightarrow\)  General) to compile the 4 feature classes into 1.  Name the output ‘streams\_tmp’

\item {} 
The Streams segments need to be connected with small single-cell length lines at the inlet/outlet locations. (If you have only frontal hucs, this step is not necessary)
\begin{itemize}
\item {} 
Start Editing with the global personal geodatabase as your target

\item {} 
Set selectable layers to ‘streams\_tmp’

\item {} 
Turn snapping on for ends

\item {} 
Zoom into the outlet of 1111. Starting at the upstream end, add a line connecting the end node outlet of 1111 with the end node inlet of  4444.

\item {} 
Repeat for 2222 and 3333 inlets.

\end{itemize}

\item {} 
Symbolize the streams\_tmp layer with arrows to see if the stream lines are pointing downstream.
\begin{itemize}
\item {} 
You may notice several arcs pointing upstream.  These need to be flipped.

\item {} 
Zoom into one.  In the editor toolbar, change the Task to Modify Feature.  Use the Edit tool from the editor toolbar.  Select the line that needs to be flipped.  Right click on that line and select flip

\item {} 
Repeat for other stream segments that need to be flipped downstream

\end{itemize}

\item {} 
Save edits.  Quit the Editor session.

\item {} 
Now you need to create nodes at each location where the streams\_tmp feature class intersects the hucpoly feature class. If you have only frontal hucs, you can skip this step and copy your streams\_temp to be the main streams feature class:
\begin{itemize}
\item {} 
In Toolbox \(\rightarrow\) Analysis Tools \(\rightarrow\) Overlay, select the Intersect tool

\item {} 
Input: streams\_tmp, hucpoly

\item {} 
Output: streams

\end{itemize}

\item {} 
Delete “streams\_tmp” and the other strm\_xxxx layers

\item {} 
Save and close ArcMap

\end{itemize}


\paragraph{\sphinxstyleemphasis{Global Geometric Network}}
\label{\detokenize{ex_2:global-geometric-network}}

\subparagraph{\sphinxstylestrong{Initial Geometric Network Creation}}
\label{\detokenize{ex_2:initial-geometric-network-creation}}
Now we will create the global geometric network \sphinxstylestrong{NOTE:} This will not work in ArcPro
\begin{itemize}
\item {} 
Open ArcCatalog

\item {} 
Right click on the Layers feature dataset.  Choose New Geometric Network. Set the name to “huc\_net” . Say yes to snapping and take the default

\end{itemize}

\begin{figure}[htbp]
\centering
\capstart

\noindent\sphinxincludegraphics{{newgeonet}.png}
\caption{Figure: Screen capture of the inputs for the Geometric Network Creation tool}\label{\detokenize{ex_2:id25}}\end{figure}
\begin{itemize}
\item {} 
Click Next. Select ‘streams’ as the feature class for building your network.  (your options may vary based on what you’ve done so far)

\end{itemize}

\begin{figure}[htbp]
\centering
\capstart

\noindent\sphinxincludegraphics{{buildfeaturenewgeonet}.png}
\caption{Figure: Screen capture of the selected features for the Geometric Network Creation tool}\label{\detokenize{ex_2:id26}}\end{figure}
\begin{itemize}
\item {} 
Answer Simple/None to complex edges.

\item {} 
You do not need to add weights to the network

\item {} 
Depending on your version, it may ask you about preserving enabled values. If so, say Yes for preserving values.

\end{itemize}

\begin{figure}[htbp]
\centering
\capstart

\noindent\sphinxincludegraphics{{preservevalues}.png}
\caption{Figure: Screen capture of the Preserve Values inputs for the Geometric Network Creation tool}\label{\detokenize{ex_2:id27}}\end{figure}
\begin{itemize}
\item {} 
Depending on your version, it may ask you whether you want to add weights. Say No for weights.

\item {} 
Make sure you like the inputs at the summary screen, and then click Finish

\end{itemize}

\begin{figure}[htbp]
\centering
\capstart

\noindent\sphinxincludegraphics{{summarygeonet}.png}
\caption{Figure: Screen capture of the Summary of Inputs screen for the Geometric Network Creation tool}\label{\detokenize{ex_2:id28}}\end{figure}
\begin{itemize}
\item {} 
Several new layers will be added to the feature dataset:

\end{itemize}

\begin{figure}[htbp]
\centering
\capstart

\noindent\sphinxincludegraphics{{geonetgdbview}.png}
\caption{Figure: View of the new files added to the feature geodatabase}\label{\detokenize{ex_2:id29}}\end{figure}


\subparagraph{\sphinxstylestrong{Enabling Geometric Network}}
\label{\detokenize{ex_2:enabling-geometric-network}}\begin{itemize}
\item {} 
Open makeglobal.mxd.
\begin{itemize}
\item {} 
Remove all layers from the legend.

\item {} 
Add the layers feature dataset, which adds all the feature classes.

\end{itemize}

\end{itemize}


\subparagraph{\sphinxstyleemphasis{Assign HydroID}}
\label{\detokenize{ex_2:assign-hydroid}}\begin{itemize}
\item {} 
Assign the HydroID for the portions of the network.
\begin{itemize}
\item {} 
From theArcHydro Attribute Tools pulldown menu, choose ‘Assign Hydro ID’

\item {} 
It will think for a moment, and then say that all HydroIDs have been assigned and give a count

\end{itemize}

\end{itemize}

\begin{figure}[htbp]
\centering
\capstart

\noindent\sphinxincludegraphics{{assignhydroid}.png}
\caption{Figure: Illustration of the location of the Assign HydroID tool on the ArcHydro toolbar}\label{\detokenize{ex_2:id30}}\end{figure}


\subparagraph{\sphinxstyleemphasis{Set Flow Direction}}
\label{\detokenize{ex_2:set-flow-direction}}\begin{itemize}
\item {} 
From the ArcHydro Network Tools pulldown menu, choose ‘Set Flow Direction.’

\end{itemize}

\begin{figure}[htbp]
\centering
\capstart

\noindent\sphinxincludegraphics{{setfdtoolbar}.png}
\caption{Figure: Illustration of the location of the Set Flow Direction tool on the ArcHydro toolbar}\label{\detokenize{ex_2:id31}}\end{figure}
\begin{itemize}
\item {} 
Select the streams layer and With Digitized Direction.
\begin{itemize}
\item {} 
Click ok.

\item {} 
It will say that it successfully completed.

\end{itemize}

\end{itemize}

\begin{figure}[htbp]
\centering
\capstart

\noindent\sphinxincludegraphics{{setfdinputs}.png}
\caption{Figure: Screen capture of the inputs for the Set Flow Direction tool}\label{\detokenize{ex_2:id32}}\end{figure}


\subparagraph{\sphinxstyleemphasis{Connect Local Folder Streams}}
\label{\detokenize{ex_2:connect-local-folder-streams}}\begin{itemize}
\item {} 
Add two fields to the hucpoly feature class:
\begin{itemize}
\item {} 
Create a long integer field called “JunctionID”

\item {} 
Also add a 10 digit-length text field and call it “Name”

\end{itemize}

\item {} 
Update the Name field for the Hucs
\begin{itemize}
\item {} 
Start an editing session

\item {} 
Populate the Name field for each huc (01091111, 01092222, 01093333, and 01094444)

\item {} 
Save your edits.

\end{itemize}

\item {} 
Coordinate the HydroID and JunctionID fields.
\begin{itemize}
\item {} 
Still in an editing session,

\item {} 
Turn on the Spatial Adjustment toolbar.

\item {} 
On the Spatial Adjustment pulldown, select Attribute Transfer Mapping

\end{itemize}

\end{itemize}

\begin{figure}[htbp]
\centering
\capstart

\noindent\sphinxincludegraphics{{attribtxfrmapping}.png}
\caption{Figure: Illustration of the location of the Attribute Transfer Mapping dialog box on the Spatial Adjustment toolbar}\label{\detokenize{ex_2:id33}}\end{figure}
\begin{itemize}
\item {} 
For Source Layer select huc\_net\_junctions and HydroID from the fields

\item {} 
For Target Layer select hucpoly and double click JunctionID from the fields
\begin{itemize}
\item {} 
Double clicking matches the fields. It won’t work unless they are both highlighted.

\item {} 
Click ok

\end{itemize}

\end{itemize}

\begin{figure}[htbp]
\centering
\capstart

\noindent\sphinxincludegraphics{{attribtxfrinputs}.png}
\caption{Figure: Screen capture of the inputs for the Attribute Transfer Mapping tool}\label{\detokenize{ex_2:id34}}\end{figure}
\begin{itemize}
\item {} 
Next, use the Attribute Transfer Tool to link the information.
\begin{itemize}
\item {} 
Zoom in so you can see the junction where the stream intersects the border between 1111 and 4444.

\item {} 
On the right end of the Spatial Adjustment Toolbar, select the Attribute Transfer Tool.

\end{itemize}

\end{itemize}

\begin{figure}[htbp]
\centering
\capstart

\noindent\sphinxincludegraphics{{spatadjtoolbar}.png}
\caption{Figure: Illustration of the location of the Attribute Transfer Mapping tool on the Spatial Adjustment toolbar}\label{\detokenize{ex_2:id35}}\end{figure}
\begin{itemize}
\item {} 
Left-click first on the junction

\end{itemize}

\begin{figure}[htbp]
\centering
\capstart

\noindent\sphinxincludegraphics{{spatadjjxn}.png}
\caption{Figure: Screen capture showing a highlighted huc\_net\_Junction point using the Attribute Transfer Mapping tool}\label{\detokenize{ex_2:id36}}\end{figure}
\begin{itemize}
\item {} 
Then left click anywhere inside the (upstream) 1111 hucpoly.

\end{itemize}

\begin{figure}[htbp]
\centering
\capstart

\noindent\sphinxincludegraphics{{spatadjupstrm}.png}
\caption{Figure: Screen capture showing the process of clicking in the upstream local processing unit using the Attribute Transfer Mapping tool}\label{\detokenize{ex_2:id37}}\end{figure}
\begin{itemize}
\item {} 
Repeat this for the other 3 Hucs.

\item {} 
Open hucpoly to see that all 4 globally-connected hucpoly features now have a JunctionID and Name.

\end{itemize}

\begin{figure}[htbp]
\centering
\capstart

\noindent\sphinxincludegraphics{{jxnidtable}.png}
\caption{Figure: A screen capture of a HucPoly attribute table with JunctionIDs attached to features}\label{\detokenize{ex_2:id38}}\end{figure}
\begin{itemize}
\item {} 
Save and Stop editing. Save makeglobal.mxd.  Quit out of ArcMap

\end{itemize}


\subparagraph{\sphinxstyleemphasis{Notes}}
\label{\detokenize{ex_2:notes}}\begin{itemize}
\item {} 
If your huc is an upstream Huc, then you click on the junction node for that Huc.

\item {} 
If it is a coastal/frontal Huc, last in a series of globally-connected Hucs, then click on the last node of the Huc streamline.

\end{itemize}

\begin{figure}[htbp]
\centering
\capstart

\noindent\sphinxincludegraphics{{spatadjcoast}.png}
\caption{Figure: Screen capture showing the process of clicking in a coastal local processing unit using the Attribute Transfer Mapping tool}\label{\detokenize{ex_2:id39}}\end{figure}
\begin{itemize}
\item {} 
If your Huc is a solitary Huc, not connected to other Hucs upstream or downstream, you do not need to do this process. It is theoretically possible to have a hucpoly that is almost entirely solitary hucs, as long as there is at least one streams feature somewhere in the hucpoly.

\end{itemize}


\subparagraph{\sphinxstylestrong{Relationship Class}}
\label{\detokenize{ex_2:relationship-class}}\begin{itemize}
\item {} 
Open ArcCatalog

\item {} 
Right click the global geodatabase and select “New \(\rightarrow\) Relationship Class”

\item {} 
Name the relationship class “HUCHasJunction”
\begin{itemize}
\item {} 
Origin Table is hucpoly

\item {} 
Destination Table is huc\_net\_Junctions

\item {} 
Click “Next”

\end{itemize}

\end{itemize}

\begin{figure}[htbp]
\centering
\capstart

\noindent\sphinxincludegraphics{{newrelshpinputs}.png}
\caption{Figure: Screen capture of the inputs for the New Relationship Class tool}\label{\detokenize{ex_2:id40}}\end{figure}
\begin{itemize}
\item {} 
Make it a simple relationship. Click “Next.”

\item {} 
Specify a label
\begin{itemize}
\item {} 
From the origin table/feature class to destination: huc\_net\_Junctions

\item {} 
From the destination table/feature class to origin: hucpoly

\item {} 
Enter ‘none’ for message direction

\end{itemize}

\end{itemize}

\begin{figure}[htbp]
\centering
\capstart

\noindent\sphinxincludegraphics{{newrelshplabel}.png}
\caption{Figure: Screen capture of the inputs for the labels tab of the New Relationship Class tool}\label{\detokenize{ex_2:id41}}\end{figure}
\begin{itemize}
\item {} 
On the next tab, select one to one cardinality

\item {} 
Do not add attributes

\item {} 
From the drop-down, pick Primary Key=JunctionID ; Foreign Key= HydroID

\end{itemize}

\begin{figure}[htbp]
\centering
\capstart

\noindent\sphinxincludegraphics{{newrelshpprimkey}.png}
\caption{Figure: Screen capture of the inputs for the pimrary key tab of the New Relationship Class tool}\label{\detokenize{ex_2:id42}}\end{figure}
\begin{itemize}
\item {} 
Click Next, then Finish

\end{itemize}


\paragraph{\sphinxstyleemphasis{Test Delineation}}
\label{\detokenize{ex_2:test-delineation}}\begin{itemize}
\item {} 
Start a fresh map document in ArcMap

\item {} 
Add the following layers to your map document:
\begin{itemize}
\item {} 
huc\_net (depending on where you are in the process, this will also add Point/Point3D, Streams/Streams3D, and huc\_net\_junctions)

\item {} 
hucpoly

\item {} 
Your str900 layer (or other str*** if you have a different catchment number/pixel size)

\item {} 
Adjoint catchment and catchment features classes

\item {} 
Fac, fdr, dem grids

\end{itemize}

\item {} 
Save your map document

\item {} 
On the ArcHydro toolbar, select the delineate button. Either the Global or Point Delineation buttons will work, but if you want to test whether your global is working you should use the Global button.

\end{itemize}

\begin{figure}[htbp]
\centering
\capstart

\noindent\sphinxincludegraphics{{delinbuttons}.png}
\caption{Figure: Illustration of the location of the watershed delineation buttons on the ArcHydro toolbar}\label{\detokenize{ex_2:id43}}\end{figure}
\begin{itemize}
\item {} 
Before you can delineate with the global button, it will ask you to specify the paths to the global data folder
\begin{itemize}
\item {} 
If you have already added the layers to the map, it will auto-populate for you

\item {} 
Add the navigation path to your archydro (global) folder)

\item {} 
Click OK

\item {} 
The layer path window will disappear, but the paths are set in the background

\end{itemize}

\end{itemize}

\begin{figure}[htbp]
\centering
\capstart

\noindent\sphinxincludegraphics{{globaldelinpaths}.png}
\caption{Figure: Screen capture of the inputs for the Global Point Delineation tool}\label{\detokenize{ex_2:id44}}\end{figure}
\begin{itemize}
\item {} 
Zoom in to the area where you want to test your delineation.

\item {} 
Click the Global delineation button again.
\begin{itemize}
\item {} 
It will ask you again for your path.

\item {} 
Click OK

\item {} 
Your cursor will turn into a crosshair/plus sign

\item {} 
Click on the pixel of your str900 grid

\end{itemize}

\item {} 
Your cursor will turn into a crosshair/plus sign.
\begin{itemize}
\item {} 
Click on the delineation point

\item {} 
AH will ask you about snapping.
\begin{itemize}
\item {} 
Say Yes to snapping

\item {} 
Snapping distance of 3 pixels

\end{itemize}

\end{itemize}

\end{itemize}

\begin{figure}[htbp]
\centering
\capstart

\noindent\sphinxincludegraphics{{snappoint}.png}
\caption{Figure: Screen capture of the inputs for the snap point settings}\label{\detokenize{ex_2:id45}}\end{figure}
\begin{itemize}
\item {} 
The click point will turn into a red dot
\begin{itemize}
\item {} 
It may think for a moment

\item {} 
Your finished watershed will be a red cross-hatched area:

\end{itemize}

\end{itemize}

\begin{figure}[htbp]
\centering
\capstart

\noindent\sphinxincludegraphics{{tempdelin}.png}
\caption{Figure: Screen capture of the delineation output}\label{\detokenize{ex_2:id46}}\end{figure}
\begin{itemize}
\item {} 
A popup will ask you if you want to save the watershed.
\begin{itemize}
\item {} 
Say yes, unless for some reason you can tell it went horribly wrong.

\item {} 
If your delineation went horribly wrong, you may want to save it too, just to troubleshoot

\end{itemize}

\item {} 
Name the watershed something descriptive, in case you are doing multiple tests. They will be stored in the file gdb that was created when you saved your map document. The file gdb will have the same name as your map document.

\end{itemize}

\sphinxstylestrong{Next Step: Do Exercise 3}


\subsection{Exercise 3}
\label{\detokenize{ex_3:exercise-3}}\label{\detokenize{ex_3::doc}}

\subsection{Exercise 4}
\label{\detokenize{ex_4:exercise-4}}\label{\detokenize{ex_4::doc}}

\subsection{First Steps}
\label{\detokenize{firstSteps:first-steps}}\label{\detokenize{firstSteps::doc}}
\noindent\sphinxincludegraphics{{doggo}.jpg}

\begin{figure}[htbp]
\centering
\capstart

\noindent\sphinxincludegraphics[height=300\sphinxpxdimen]{{osabear}.JPG}
\caption{\sphinxstylestrong{Figure 1:} A regal Chesapeake Bay Retriever}\label{\detokenize{firstSteps:id1}}\end{figure}


\section{StreamStats\_DataPrep ESRI Toolbox}
\label{\detokenize{StreamStats_DataPrep:streamstats-dataprep-esri-toolbox}}\label{\detokenize{StreamStats_DataPrep::doc}}

\subsection{Database Setup}
\label{\detokenize{StreamStats_DataPrep:database-setup}}\index{databaseSetup (class in StreamStats\_DataPrep)@\spxentry{databaseSetup}\spxextra{class in StreamStats\_DataPrep}}

\begin{fulllineitems}
\phantomsection\label{\detokenize{StreamStats_DataPrep:StreamStats_DataPrep.databaseSetup}}\pysigline{\sphinxbfcode{\sphinxupquote{class }}\sphinxcode{\sphinxupquote{StreamStats\_DataPrep.}}\sphinxbfcode{\sphinxupquote{databaseSetup}}}
Set up the workspace needed to process elevation and hydrography data.

This tool is a wrapper on {\hyperref[\detokenize{databaseSetup:databaseSetup.databaseSetup}]{\sphinxcrossref{\sphinxcode{\sphinxupquote{databaseSetup.databaseSetup()}}}}}.
\subsubsection*{Methods}


\begin{savenotes}\sphinxatlongtablestart\begin{longtable}{\X{1}{2}\X{1}{2}}
\hline

\endfirsthead

\multicolumn{2}{c}%
{\makebox[0pt]{\sphinxtablecontinued{\tablename\ \thetable{} -- continued from previous page}}}\\
\hline

\endhead

\hline
\multicolumn{2}{r}{\makebox[0pt][r]{\sphinxtablecontinued{Continued on next page}}}\\
\endfoot

\endlastfoot

{\hyperref[\detokenize{StreamStats_DataPrep:StreamStats_DataPrep.databaseSetup.getParameterInfo}]{\sphinxcrossref{\sphinxcode{\sphinxupquote{getParameterInfo}}}}}(self)
&
Database Setup inputs.
\\
\hline
\end{longtable}\sphinxatlongtableend\end{savenotes}
\index{getParameterInfo() (StreamStats\_DataPrep.databaseSetup method)@\spxentry{getParameterInfo()}\spxextra{StreamStats\_DataPrep.databaseSetup method}}

\begin{fulllineitems}
\phantomsection\label{\detokenize{StreamStats_DataPrep:StreamStats_DataPrep.databaseSetup.getParameterInfo}}\pysiglinewithargsret{\sphinxbfcode{\sphinxupquote{getParameterInfo}}}{\emph{self}}{}
Database Setup inputs.
\begin{quote}\begin{description}
\item[{Parameters}] \leavevmode\begin{description}
\item[{\sphinxstylestrong{Output Workspace}}] \leavevmode{[}DEWorkspace (File System){]}
Folder-type workspace for local folders and geodatabase to be created.

\item[{\sphinxstylestrong{Main ArcHydro Geodatabase Name}}] \leavevmode{[}GPString{]}
Name of the geodatabase to be created in “Output Workspace.”

\item[{\sphinxstylestrong{Hydrologic Unit Boundary Dataset}}] \leavevmode{[}DEShapefile or DEFeatureClass{]}
Polygon vector defining local processing units. Should have columns for outwalls and inwalls, see below.

\item[{\sphinxstylestrong{Outwall Field}}] \leavevmode{[}Field{]}
Field in “Hydrologic Unit Boundary Dataset” used to determine local folders and outwalls.

\item[{\sphinxstylestrong{Inwall Field}}] \leavevmode{[}Field{]}
Field in “Hydrologic Unit Boundary Dataset” used to determine inwalls.

\item[{\sphinxstylestrong{Hydrologic Unit Buffer Distance (m)}}] \leavevmode{[}GPString{]}
Distance to buffer local folder polygons by.

\item[{\sphinxstylestrong{Input Hydrography Workspace}}] \leavevmode{[}DEWorkspace{]}
Path to folder type workspace with geodatabases of NHD hydrography.

\item[{\sphinxstylestrong{Elevation Dataset Template}}] \leavevmode{[}DERasterBand{]}
Raster dataset to pull projection information from, works best as an ESRI grid.

\item[{\sphinxstylestrong{Alternative Outwall Buffer}}] \leavevmode{[}GPString (optional){]}
Distance for alternative outwall buffer.

\end{description}

\item[{Returns}] \leavevmode\begin{description}
\item[{\sphinxstylestrong{parameters}}] \leavevmode{[}list{]}
List of input parameters passed to the execute method.

\end{description}

\end{description}\end{quote}

\end{fulllineitems}


\end{fulllineitems}



\subsection{Make Elevation Data Index}
\label{\detokenize{StreamStats_DataPrep:make-elevation-data-index}}\index{makeELEVDATAIndex (class in StreamStats\_DataPrep)@\spxentry{makeELEVDATAIndex}\spxextra{class in StreamStats\_DataPrep}}

\begin{fulllineitems}
\phantomsection\label{\detokenize{StreamStats_DataPrep:StreamStats_DataPrep.makeELEVDATAIndex}}\pysigline{\sphinxbfcode{\sphinxupquote{class }}\sphinxcode{\sphinxupquote{StreamStats\_DataPrep.}}\sphinxbfcode{\sphinxupquote{makeELEVDATAIndex}}}
Create a seamless raster mosaic dataset from input digital elevation tiles.

This tool is a wrapper on {\hyperref[\detokenize{elevationTools:elevationTools.elevIndex}]{\sphinxcrossref{\sphinxcode{\sphinxupquote{elevationTools.elevIndex()}}}}}.
\subsubsection*{Methods}


\begin{savenotes}\sphinxatlongtablestart\begin{longtable}{\X{1}{2}\X{1}{2}}
\hline

\endfirsthead

\multicolumn{2}{c}%
{\makebox[0pt]{\sphinxtablecontinued{\tablename\ \thetable{} -- continued from previous page}}}\\
\hline

\endhead

\hline
\multicolumn{2}{r}{\makebox[0pt][r]{\sphinxtablecontinued{Continued on next page}}}\\
\endfoot

\endlastfoot

{\hyperref[\detokenize{StreamStats_DataPrep:StreamStats_DataPrep.makeELEVDATAIndex.getParameterInfo}]{\sphinxcrossref{\sphinxcode{\sphinxupquote{getParameterInfo}}}}}(self)
&
Make ELEV data index inputs
\\
\hline
\end{longtable}\sphinxatlongtableend\end{savenotes}
\index{getParameterInfo() (StreamStats\_DataPrep.makeELEVDATAIndex method)@\spxentry{getParameterInfo()}\spxextra{StreamStats\_DataPrep.makeELEVDATAIndex method}}

\begin{fulllineitems}
\phantomsection\label{\detokenize{StreamStats_DataPrep:StreamStats_DataPrep.makeELEVDATAIndex.getParameterInfo}}\pysiglinewithargsret{\sphinxbfcode{\sphinxupquote{getParameterInfo}}}{\emph{self}}{}
Make ELEV data index inputs
\begin{quote}\begin{description}
\item[{Parameters}] \leavevmode\begin{description}
\item[{\sphinxstylestrong{Output Geodatabase}}] \leavevmode{[}DEWorkspace (Geodatabase){]}
Path to the geodatabase that will hold the output raster mosaic dataset.

\item[{\sphinxstylestrong{Output Raster Mosaic Dataset Name}}] \leavevmode{[}GPString{]}
Name of raster mosaic dataset to output, defaults to IndexRMD.

\item[{\sphinxstylestrong{Coordinate System}}] \leavevmode{[}GPCoordinateSystem{]}
Coordinate system of input grids and raster mosaic dataset.

\item[{\sphinxstylestrong{Input Elevation Data workspace}}] \leavevmode{[}DEWorkspace (Folder){]}
Path to folder holding input digital elevation models to be included in the raster mosaic dataset.

\end{description}

\item[{Returns}] \leavevmode\begin{description}
\item[{\sphinxstylestrong{parameters}}] \leavevmode{[}list{]}
List of input parameters passed to the execute method.

\end{description}

\end{description}\end{quote}

\end{fulllineitems}


\end{fulllineitems}



\subsection{Extract Polygons}
\label{\detokenize{StreamStats_DataPrep:extract-polygons}}\index{ExtractPoly (class in StreamStats\_DataPrep)@\spxentry{ExtractPoly}\spxextra{class in StreamStats\_DataPrep}}

\begin{fulllineitems}
\phantomsection\label{\detokenize{StreamStats_DataPrep:StreamStats_DataPrep.ExtractPoly}}\pysigline{\sphinxbfcode{\sphinxupquote{class }}\sphinxcode{\sphinxupquote{StreamStats\_DataPrep.}}\sphinxbfcode{\sphinxupquote{ExtractPoly}}}
Extract a hydrologic unit from a digital elevation model based on a clipping polygon.

This tool is a wrapper on {\hyperref[\detokenize{elevationTools:elevationTools.extractPoly}]{\sphinxcrossref{\sphinxcode{\sphinxupquote{elevationTools.extractPoly()}}}}}.
\subsubsection*{Methods}


\begin{savenotes}\sphinxatlongtablestart\begin{longtable}{\X{1}{2}\X{1}{2}}
\hline

\endfirsthead

\multicolumn{2}{c}%
{\makebox[0pt]{\sphinxtablecontinued{\tablename\ \thetable{} -- continued from previous page}}}\\
\hline

\endhead

\hline
\multicolumn{2}{r}{\makebox[0pt][r]{\sphinxtablecontinued{Continued on next page}}}\\
\endfoot

\endlastfoot

{\hyperref[\detokenize{StreamStats_DataPrep:StreamStats_DataPrep.ExtractPoly.getParameterInfo}]{\sphinxcrossref{\sphinxcode{\sphinxupquote{getParameterInfo}}}}}(self)
&
Extract Polygon inputs.
\\
\hline
\end{longtable}\sphinxatlongtableend\end{savenotes}
\index{getParameterInfo() (StreamStats\_DataPrep.ExtractPoly method)@\spxentry{getParameterInfo()}\spxextra{StreamStats\_DataPrep.ExtractPoly method}}

\begin{fulllineitems}
\phantomsection\label{\detokenize{StreamStats_DataPrep:StreamStats_DataPrep.ExtractPoly.getParameterInfo}}\pysiglinewithargsret{\sphinxbfcode{\sphinxupquote{getParameterInfo}}}{\emph{self}}{}
Extract Polygon inputs.
\begin{quote}\begin{description}
\item[{Parameters}] \leavevmode\begin{description}
\item[{\sphinxstylestrong{Output Workspace}}] \leavevmode{[}DEWorkspace (Folder){]}
Path to folder to work in.

\item[{\sphinxstylestrong{ELEVDATA Raster Mosaic Dataset}}] \leavevmode{[}DEMosaicDataset{]}
Path to the raster mosaic dataset holding the elevation data.

\item[{\sphinxstylestrong{Clip Polygon}}] \leavevmode{[}GPFeatureLayer{]}
Feature class of the watershed boundary being used for clipping.

\item[{\sphinxstylestrong{Output Grid}}] \leavevmode{[}GPString{]}
Name of the output ESRI grid, defaults to dem\_dd.

\end{description}

\item[{Returns}] \leavevmode\begin{description}
\item[{\sphinxstylestrong{parameters}}] \leavevmode{[}list{]}
List of input parameters passed to the execute method.

\end{description}

\end{description}\end{quote}

\end{fulllineitems}


\end{fulllineitems}



\subsection{Check For NoData Cells}
\label{\detokenize{StreamStats_DataPrep:check-for-nodata-cells}}\index{CheckNoData (class in StreamStats\_DataPrep)@\spxentry{CheckNoData}\spxextra{class in StreamStats\_DataPrep}}

\begin{fulllineitems}
\phantomsection\label{\detokenize{StreamStats_DataPrep:StreamStats_DataPrep.CheckNoData}}\pysigline{\sphinxbfcode{\sphinxupquote{class }}\sphinxcode{\sphinxupquote{StreamStats\_DataPrep.}}\sphinxbfcode{\sphinxupquote{CheckNoData}}}
Check for no data cells in a digital elevation model.

This tool is a wrapper on {\hyperref[\detokenize{elevationTools:elevationTools.checkNoData}]{\sphinxcrossref{\sphinxcode{\sphinxupquote{elevationTools.checkNoData()}}}}}.
\subsubsection*{Methods}


\begin{savenotes}\sphinxatlongtablestart\begin{longtable}{\X{1}{2}\X{1}{2}}
\hline

\endfirsthead

\multicolumn{2}{c}%
{\makebox[0pt]{\sphinxtablecontinued{\tablename\ \thetable{} -- continued from previous page}}}\\
\hline

\endhead

\hline
\multicolumn{2}{r}{\makebox[0pt][r]{\sphinxtablecontinued{Continued on next page}}}\\
\endfoot

\endlastfoot

{\hyperref[\detokenize{StreamStats_DataPrep:StreamStats_DataPrep.CheckNoData.getParameterInfo}]{\sphinxcrossref{\sphinxcode{\sphinxupquote{getParameterInfo}}}}}(self)
&
Check for no data inputs.
\\
\hline
\end{longtable}\sphinxatlongtableend\end{savenotes}
\index{getParameterInfo() (StreamStats\_DataPrep.CheckNoData method)@\spxentry{getParameterInfo()}\spxextra{StreamStats\_DataPrep.CheckNoData method}}

\begin{fulllineitems}
\phantomsection\label{\detokenize{StreamStats_DataPrep:StreamStats_DataPrep.CheckNoData.getParameterInfo}}\pysiglinewithargsret{\sphinxbfcode{\sphinxupquote{getParameterInfo}}}{\emph{self}}{}
Check for no data inputs.
\begin{quote}\begin{description}
\item[{Parameters}] \leavevmode\begin{description}
\item[{\sphinxstylestrong{InputGrid}}] \leavevmode{[}DERasterBand{]}
Path to raster dataset to examine.

\item[{\sphinxstylestrong{Workspace}}] \leavevmode{[}DEWorkspace (Geodatabase){]}
Geodatabase-type workspace to work in.

\item[{\sphinxstylestrong{Output Feature Layer}}] \leavevmode{[}GPString{]}
Name of output feature class, defaults to DEM\_NoDataSinks.

\end{description}

\item[{Returns}] \leavevmode\begin{description}
\item[{\sphinxstylestrong{parameters}}] \leavevmode{[}list{]}
List of input parameters passed to the execute method.

\end{description}

\end{description}\end{quote}

\end{fulllineitems}


\end{fulllineitems}



\subsection{Fill NoData Cells}
\label{\detokenize{StreamStats_DataPrep:fill-nodata-cells}}\index{FillNoData (class in StreamStats\_DataPrep)@\spxentry{FillNoData}\spxextra{class in StreamStats\_DataPrep}}

\begin{fulllineitems}
\phantomsection\label{\detokenize{StreamStats_DataPrep:StreamStats_DataPrep.FillNoData}}\pysigline{\sphinxbfcode{\sphinxupquote{class }}\sphinxcode{\sphinxupquote{StreamStats\_DataPrep.}}\sphinxbfcode{\sphinxupquote{FillNoData}}}
Fill no data cells in a digital elevation model.

This tool is a wrapper on {\hyperref[\detokenize{elevationTools:elevationTools.fillNoData}]{\sphinxcrossref{\sphinxcode{\sphinxupquote{elevationTools.fillNoData()}}}}}.
\subsubsection*{Notes}

This tool can be run iteratively to fully fill no data areas that are larger than one cell.
\subsubsection*{Methods}


\begin{savenotes}\sphinxatlongtablestart\begin{longtable}{\X{1}{2}\X{1}{2}}
\hline

\endfirsthead

\multicolumn{2}{c}%
{\makebox[0pt]{\sphinxtablecontinued{\tablename\ \thetable{} -- continued from previous page}}}\\
\hline

\endhead

\hline
\multicolumn{2}{r}{\makebox[0pt][r]{\sphinxtablecontinued{Continued on next page}}}\\
\endfoot

\endlastfoot

{\hyperref[\detokenize{StreamStats_DataPrep:StreamStats_DataPrep.FillNoData.getParameterInfo}]{\sphinxcrossref{\sphinxcode{\sphinxupquote{getParameterInfo}}}}}(self)
&
Fill no data inputs.
\\
\hline
\end{longtable}\sphinxatlongtableend\end{savenotes}
\index{getParameterInfo() (StreamStats\_DataPrep.FillNoData method)@\spxentry{getParameterInfo()}\spxextra{StreamStats\_DataPrep.FillNoData method}}

\begin{fulllineitems}
\phantomsection\label{\detokenize{StreamStats_DataPrep:StreamStats_DataPrep.FillNoData.getParameterInfo}}\pysiglinewithargsret{\sphinxbfcode{\sphinxupquote{getParameterInfo}}}{\emph{self}}{}
Fill no data inputs.
\begin{quote}\begin{description}
\item[{Parameters}] \leavevmode\begin{description}
\item[{\sphinxstylestrong{Workspace}}] \leavevmode{[}DEWorkspace (Folder){]}
Path to folder to work in.

\item[{\sphinxstylestrong{Input Grid}}] \leavevmode{[}DERasterBand{]}
Path to raster dataset with no data values to be filled, defaults to DEM\_NoDataSinks.

\item[{\sphinxstylestrong{Output Grid}}] \leavevmode{[}GPString{]}
Path to write out filled raster dataset to, defaults to DEM\_filled.

\end{description}

\item[{Returns}] \leavevmode\begin{description}
\item[{\sphinxstylestrong{parameters}}] \leavevmode{[}list{]}
List of input parameters passed to the execute method.

\end{description}

\end{description}\end{quote}

\end{fulllineitems}


\end{fulllineitems}



\subsection{Project and Scale}
\label{\detokenize{StreamStats_DataPrep:project-and-scale}}\index{ProjScale (class in StreamStats\_DataPrep)@\spxentry{ProjScale}\spxextra{class in StreamStats\_DataPrep}}

\begin{fulllineitems}
\phantomsection\label{\detokenize{StreamStats_DataPrep:StreamStats_DataPrep.ProjScale}}\pysigline{\sphinxbfcode{\sphinxupquote{class }}\sphinxcode{\sphinxupquote{StreamStats\_DataPrep.}}\sphinxbfcode{\sphinxupquote{ProjScale}}}
Project and scale a digital elevation model.

This tool is a wrapper on {\hyperref[\detokenize{elevationTools:elevationTools.projScale}]{\sphinxcrossref{\sphinxcode{\sphinxupquote{elevationTools.projScale()}}}}}.
\subsubsection*{Notes}

After scaling, this tool attempts to set the correct z-units; however, if you vertical units are different from your horizontal units it is advized to check the z-units manually.
\subsubsection*{Methods}


\begin{savenotes}\sphinxatlongtablestart\begin{longtable}{\X{1}{2}\X{1}{2}}
\hline

\endfirsthead

\multicolumn{2}{c}%
{\makebox[0pt]{\sphinxtablecontinued{\tablename\ \thetable{} -- continued from previous page}}}\\
\hline

\endhead

\hline
\multicolumn{2}{r}{\makebox[0pt][r]{\sphinxtablecontinued{Continued on next page}}}\\
\endfoot

\endlastfoot

{\hyperref[\detokenize{StreamStats_DataPrep:StreamStats_DataPrep.ProjScale.getParameterInfo}]{\sphinxcrossref{\sphinxcode{\sphinxupquote{getParameterInfo}}}}}(self)
&
Project and scale digital elevation model inputs.
\\
\hline
\end{longtable}\sphinxatlongtableend\end{savenotes}
\index{getParameterInfo() (StreamStats\_DataPrep.ProjScale method)@\spxentry{getParameterInfo()}\spxextra{StreamStats\_DataPrep.ProjScale method}}

\begin{fulllineitems}
\phantomsection\label{\detokenize{StreamStats_DataPrep:StreamStats_DataPrep.ProjScale.getParameterInfo}}\pysiglinewithargsret{\sphinxbfcode{\sphinxupquote{getParameterInfo}}}{\emph{self}}{}
Project and scale digital elevation model inputs.
\begin{quote}\begin{description}
\item[{Parameters}] \leavevmode\begin{description}
\item[{\sphinxstylestrong{InWorkSpace}}] \leavevmode{[}DEWorkspace (Folder){]}
Path to the folder to work in.

\item[{\sphinxstylestrong{InGrid}}] \leavevmode{[}DERasterBand{]}
Path to the raster dataset to project and scale.

\item[{\sphinxstylestrong{OutGrid}}] \leavevmode{[}GPString{]}
Name for the projected and scaled raster, defaults to dem\_raw.

\item[{\sphinxstylestrong{OutCoordSys}}] \leavevmode{[}GPCoordinateSystem{]}
Coordinate system to project the input raster to.

\item[{\sphinxstylestrong{OutCellSize}}] \leavevmode{[}analysis\_cell\_size{]}
Output cell size to project the input raster to, defaults to 10 horizontal map units.

\item[{\sphinxstylestrong{RegPt}}] \leavevmode{[}GPString{]}
Registration point for the projected raster, defaults to “0 0”.

\item[{\sphinxstylestrong{scaleFact}}] \leavevmode{[}GPString{]}
Scale factor to use to convert the projected raster to integers, defaults to 100. Consider using a larger scale factor as cell-size decreases.

\end{description}

\item[{Returns}] \leavevmode\begin{description}
\item[{\sphinxstylestrong{parameters}}] \leavevmode{[}list{]}
List of input parameters passed to the execute method.

\end{description}

\end{description}\end{quote}

\end{fulllineitems}


\end{fulllineitems}



\subsection{TopoGrid (optional)}
\label{\detokenize{StreamStats_DataPrep:topogrid-optional}}\index{TopoGrid (class in StreamStats\_DataPrep)@\spxentry{TopoGrid}\spxextra{class in StreamStats\_DataPrep}}

\begin{fulllineitems}
\phantomsection\label{\detokenize{StreamStats_DataPrep:StreamStats_DataPrep.TopoGrid}}\pysigline{\sphinxbfcode{\sphinxupquote{class }}\sphinxcode{\sphinxupquote{StreamStats\_DataPrep.}}\sphinxbfcode{\sphinxupquote{TopoGrid}}}
Condition an input DEM using a flowline dendrite prior to hydro-enforcement.

This tool is a wrapper on {\hyperref[\detokenize{topo_grid:topo_grid.topogrid}]{\sphinxcrossref{\sphinxcode{\sphinxupquote{topo\_grid.topogrid()}}}}}.
\subsubsection*{Notes}

This function turns the input DEM into a 3D point cloud, thinned using the VIP algorithm so that not all points are retained from the original DEM. The point cloud is used in conjunction with the supplied flowlines to re-interpolate a DEM that is aware of the location of the flowlines and their flow direction.

This is a computationally intensive function. Running it via ArcPro or Python 3 will be faster than using ArcMap or Python 2.
\subsubsection*{Methods}


\begin{savenotes}\sphinxatlongtablestart\begin{longtable}{\X{1}{2}\X{1}{2}}
\hline

\endfirsthead

\multicolumn{2}{c}%
{\makebox[0pt]{\sphinxtablecontinued{\tablename\ \thetable{} -- continued from previous page}}}\\
\hline

\endhead

\hline
\multicolumn{2}{r}{\makebox[0pt][r]{\sphinxtablecontinued{Continued on next page}}}\\
\endfoot

\endlastfoot

{\hyperref[\detokenize{StreamStats_DataPrep:StreamStats_DataPrep.TopoGrid.getParameterInfo}]{\sphinxcrossref{\sphinxcode{\sphinxupquote{getParameterInfo}}}}}(self)
&
TopoGrid inputs.
\\
\hline
\end{longtable}\sphinxatlongtableend\end{savenotes}
\index{getParameterInfo() (StreamStats\_DataPrep.TopoGrid method)@\spxentry{getParameterInfo()}\spxextra{StreamStats\_DataPrep.TopoGrid method}}

\begin{fulllineitems}
\phantomsection\label{\detokenize{StreamStats_DataPrep:StreamStats_DataPrep.TopoGrid.getParameterInfo}}\pysiglinewithargsret{\sphinxbfcode{\sphinxupquote{getParameterInfo}}}{\emph{self}}{}
TopoGrid inputs.
\begin{quote}\begin{description}
\item[{Parameters}] \leavevmode\begin{description}
\item[{\sphinxstylestrong{Output Workspace}}] \leavevmode{[}DEWorkspace (Geodatabase){]}
Path to a geodatabase to work in.

\item[{\sphinxstylestrong{Dissolved HUC8 boundary}}] \leavevmode{[}DEFeatureClass or DEShapefile{]}
Feature class to use in bounding the topogrid conditioning process.

\item[{\sphinxstylestrong{Topogrid Buffer Distance}}] \leavevmode{[}GPDouble{]}
Distance to buffer the input HUC8 boundary, in the units of the HUC8 boundary.

\item[{\sphinxstylestrong{12 Digit Hydrologic Unit Datasets if dissolved HUC8 boundary failed}}] \leavevmode{[}DEFeatureClass or DEShapefile (Optional){]}
List of HUC12 boundaries to split up TopoGrid computations if the whole domain fails.

\item[{\sphinxstylestrong{Dendritic Flowline Features}}] \leavevmode{[}DEFeatureClass or DEShapefile{]}
Dendrite used for enforcing flow direction in topogrid.

\item[{\sphinxstylestrong{Buffered and Projected Elevation Data}}] \leavevmode{[}DERasterBand or DERasterDataset{]}
Input digital elevation model to be conditioned using topogrid.

\item[{\sphinxstylestrong{Output Cell Size}}] \leavevmode{[}GPString{]}
Cell size for output digital elevation model, defualts to 10 horizontal map units.

\item[{\sphinxstylestrong{VIP Percentage}}] \leavevmode{[}GPString{]}
Thinning value used in the Very Important Points algorithm to decide how many points from the original raster are retained, defaults to 5 percent.

\item[{\sphinxstylestrong{SnapGrid}}] \leavevmode{[}DERasterBand (Optional){]}
Raster to snap output grid to.

\end{description}

\item[{Returns}] \leavevmode\begin{description}
\item[{\sphinxstylestrong{parameters}}] \leavevmode{[}list{]}
List of input parameters passed to the execute method.

\end{description}

\end{description}\end{quote}

\end{fulllineitems}


\end{fulllineitems}



\subsection{Bathymetic Gradient Setup}
\label{\detokenize{StreamStats_DataPrep:bathymetic-gradient-setup}}\index{SetupBathyGrad (class in StreamStats\_DataPrep)@\spxentry{SetupBathyGrad}\spxextra{class in StreamStats\_DataPrep}}

\begin{fulllineitems}
\phantomsection\label{\detokenize{StreamStats_DataPrep:StreamStats_DataPrep.SetupBathyGrad}}\pysigline{\sphinxbfcode{\sphinxupquote{class }}\sphinxcode{\sphinxupquote{StreamStats\_DataPrep.}}\sphinxbfcode{\sphinxupquote{SetupBathyGrad}}}
Prepare bathymetric gradient inputs for use in hydro-enforcement.

This tool is a wrapper on {\hyperref[\detokenize{make_hydrodem:make_hydrodem.bathymetricGradient}]{\sphinxcrossref{\sphinxcode{\sphinxupquote{make\_hydrodem.bathymetricGradient()}}}}}.
\subsubsection*{Notes}

The bathymetric gradient refers to generating a sloping area around the flowline dendrite that ensures the lanscape around the dendrite flows to the stream. This also adds a sloping surface to double-line streams and waterbodies to help insure proper drainage after hydro-enforcement.
\subsubsection*{Methods}


\begin{savenotes}\sphinxatlongtablestart\begin{longtable}{\X{1}{2}\X{1}{2}}
\hline

\endfirsthead

\multicolumn{2}{c}%
{\makebox[0pt]{\sphinxtablecontinued{\tablename\ \thetable{} -- continued from previous page}}}\\
\hline

\endhead

\hline
\multicolumn{2}{r}{\makebox[0pt][r]{\sphinxtablecontinued{Continued on next page}}}\\
\endfoot

\endlastfoot

{\hyperref[\detokenize{StreamStats_DataPrep:StreamStats_DataPrep.SetupBathyGrad.getParameterInfo}]{\sphinxcrossref{\sphinxcode{\sphinxupquote{getParameterInfo}}}}}(self)
&
Setup Bathymetric Gradient inputs.
\\
\hline
\end{longtable}\sphinxatlongtableend\end{savenotes}
\index{getParameterInfo() (StreamStats\_DataPrep.SetupBathyGrad method)@\spxentry{getParameterInfo()}\spxextra{StreamStats\_DataPrep.SetupBathyGrad method}}

\begin{fulllineitems}
\phantomsection\label{\detokenize{StreamStats_DataPrep:StreamStats_DataPrep.SetupBathyGrad.getParameterInfo}}\pysiglinewithargsret{\sphinxbfcode{\sphinxupquote{getParameterInfo}}}{\emph{self}}{}
Setup Bathymetric Gradient inputs.
\begin{quote}\begin{description}
\item[{Parameters}] \leavevmode\begin{description}
\item[{\sphinxstylestrong{Output Workspace}}] \leavevmode{[}DEWorkspace (Geodatabase){]}
Path to a geodatabase to work in.

\item[{\sphinxstylestrong{Digital Elevation Model (used for snapping)}}] \leavevmode{[}DERasterBand{]}
Path to a digital elevation model to use for aligning output grids to the rest of the project.

\item[{\sphinxstylestrong{Dissolved HUC8 Dataset}}] \leavevmode{[}DEFeatureClass{]}
Feature class of the local folder boundary.

\item[{\sphinxstylestrong{NHD Area}}] \leavevmode{[}DEFeatureClass{]}
Feature class of NHD double line streams.

\item[{\sphinxstylestrong{NHD Dendrite}}] \leavevmode{[}DEFeatureClass{]}
Feature class of the flowline dendrite.

\item[{\sphinxstylestrong{NHD Waterbody}}] \leavevmode{[}DEFeatureClass{]}
Feature class of the NHD water bodies.

\item[{\sphinxstylestrong{Cell Size}}] \leavevmode{[}GPString{]}
Output grid cell size, defaults to 10 horizontal map units.

\end{description}

\item[{Returns}] \leavevmode\begin{description}
\item[{\sphinxstylestrong{parameters}}] \leavevmode{[}list{]}
List of input parameters passed to the execute method.

\end{description}

\end{description}\end{quote}

\end{fulllineitems}


\end{fulllineitems}



\subsection{Coastal DEM Processing (optional)}
\label{\detokenize{StreamStats_DataPrep:coastal-dem-processing-optional}}\index{CoastalDEM (class in StreamStats\_DataPrep)@\spxentry{CoastalDEM}\spxextra{class in StreamStats\_DataPrep}}

\begin{fulllineitems}
\phantomsection\label{\detokenize{StreamStats_DataPrep:StreamStats_DataPrep.CoastalDEM}}\pysigline{\sphinxbfcode{\sphinxupquote{class }}\sphinxcode{\sphinxupquote{StreamStats\_DataPrep.}}\sphinxbfcode{\sphinxupquote{CoastalDEM}}}
Prepare coastal areas for hydro-enforcement.

This tool is a wrapper on {\hyperref[\detokenize{make_hydrodem:make_hydrodem.coastaldem}]{\sphinxcrossref{\sphinxcode{\sphinxupquote{make\_hydrodem.coastaldem()}}}}}.
\subsubsection*{Methods}


\begin{savenotes}\sphinxatlongtablestart\begin{longtable}{\X{1}{2}\X{1}{2}}
\hline

\endfirsthead

\multicolumn{2}{c}%
{\makebox[0pt]{\sphinxtablecontinued{\tablename\ \thetable{} -- continued from previous page}}}\\
\hline

\endhead

\hline
\multicolumn{2}{r}{\makebox[0pt][r]{\sphinxtablecontinued{Continued on next page}}}\\
\endfoot

\endlastfoot

{\hyperref[\detokenize{StreamStats_DataPrep:StreamStats_DataPrep.CoastalDEM.getParameterInfo}]{\sphinxcrossref{\sphinxcode{\sphinxupquote{getParameterInfo}}}}}(self)
&
Coastal digital elevation model processing inputs.
\\
\hline
\end{longtable}\sphinxatlongtableend\end{savenotes}
\index{getParameterInfo() (StreamStats\_DataPrep.CoastalDEM method)@\spxentry{getParameterInfo()}\spxextra{StreamStats\_DataPrep.CoastalDEM method}}

\begin{fulllineitems}
\phantomsection\label{\detokenize{StreamStats_DataPrep:StreamStats_DataPrep.CoastalDEM.getParameterInfo}}\pysiglinewithargsret{\sphinxbfcode{\sphinxupquote{getParameterInfo}}}{\emph{self}}{}
Coastal digital elevation model processing inputs.
\begin{quote}\begin{description}
\item[{Parameters}] \leavevmode\begin{description}
\item[{\sphinxstylestrong{Workspace}}] \leavevmode{[}DEWorkspace (Folder){]}
Path to folder to work in.

\item[{\sphinxstylestrong{Input raw DEM}}] \leavevmode{[}DERasterBand{]}
Original digital elevation model to be corrected for coastal areas, defaults to dem\_raw.

\item[{\sphinxstylestrong{Input LandSea polygon feature class}}] \leavevmode{[}DEFeatureClass{]}
Feature class indicating areas of land and sea.

\item[{\sphinxstylestrong{Output DEM}}] \leavevmode{[}DERasterBand{]}
Output digital elevation model name, defaults to dem\_sea.

\item[{\sphinxstylestrong{Sea Level}}] \leavevmode{[}GPString{]}
Value to insert into areas identified as the sea, defaults to -60000 vertical map units.

\end{description}

\item[{Returns}] \leavevmode\begin{description}
\item[{\sphinxstylestrong{parameters}}] \leavevmode{[}list{]}
List of input parameters passed to the execute method.

\end{description}

\end{description}\end{quote}

\end{fulllineitems}


\end{fulllineitems}



\subsection{Hydro-Enforce DEM}
\label{\detokenize{StreamStats_DataPrep:hydro-enforce-dem}}\index{HydroDEM (class in StreamStats\_DataPrep)@\spxentry{HydroDEM}\spxextra{class in StreamStats\_DataPrep}}

\begin{fulllineitems}
\phantomsection\label{\detokenize{StreamStats_DataPrep:StreamStats_DataPrep.HydroDEM}}\pysigline{\sphinxbfcode{\sphinxupquote{class }}\sphinxcode{\sphinxupquote{StreamStats\_DataPrep.}}\sphinxbfcode{\sphinxupquote{HydroDEM}}}
Hydro-Enforce a DEM.

This tool is a wrapper on {\hyperref[\detokenize{make_hydrodem:make_hydrodem.hydrodem}]{\sphinxcrossref{\sphinxcode{\sphinxupquote{make\_hydrodem.hydrodem()}}}}}.
\subsubsection*{Notes}

We suggest that AGREE defaults not be changed as this can lead to alignment issues between the flowlines and the resultant hydro-enforced DEM and subsequent products (FDR and FAC).
\subsubsection*{Methods}


\begin{savenotes}\sphinxatlongtablestart\begin{longtable}{\X{1}{2}\X{1}{2}}
\hline

\endfirsthead

\multicolumn{2}{c}%
{\makebox[0pt]{\sphinxtablecontinued{\tablename\ \thetable{} -- continued from previous page}}}\\
\hline

\endhead

\hline
\multicolumn{2}{r}{\makebox[0pt][r]{\sphinxtablecontinued{Continued on next page}}}\\
\endfoot

\endlastfoot

{\hyperref[\detokenize{StreamStats_DataPrep:StreamStats_DataPrep.HydroDEM.getParameterInfo}]{\sphinxcrossref{\sphinxcode{\sphinxupquote{getParameterInfo}}}}}(self)
&
\begin{quote}\begin{description}
\item[{Parameters}] \leavevmode
\end{description}\end{quote}

\\
\hline
\end{longtable}\sphinxatlongtableend\end{savenotes}
\index{getParameterInfo() (StreamStats\_DataPrep.HydroDEM method)@\spxentry{getParameterInfo()}\spxextra{StreamStats\_DataPrep.HydroDEM method}}

\begin{fulllineitems}
\phantomsection\label{\detokenize{StreamStats_DataPrep:StreamStats_DataPrep.HydroDEM.getParameterInfo}}\pysiglinewithargsret{\sphinxbfcode{\sphinxupquote{getParameterInfo}}}{\emph{self}}{}~\begin{quote}\begin{description}
\item[{Parameters}] \leavevmode\begin{description}
\item[{\sphinxstylestrong{Output Workspace}}] \leavevmode{[}DEWorkspace (geodatabase){]}
Geodatabase-type workspace where output raster will be saved.

\item[{\sphinxstylestrong{Scratch Workspace}}] \leavevmode{[}DEWorkspace (folder){]}
Folder-type scratch workspace.

\item[{\sphinxstylestrong{HUC layer}}] \leavevmode{[}DEFeatureClass{]}
Polygon feature class the bounds the local folder you are working in.

\item[{\sphinxstylestrong{Digital Elevation Model}}] \leavevmode{[}DERasterBand{]}
Digital elevation model to be hydro-enforced.

\item[{\sphinxstylestrong{Stream Dendrite}}] \leavevmode{[}DEFeatureClass{]}
Polyline feature class describing where streams are on the landscape.

\item[{\sphinxstylestrong{Snap Grid}}] \leavevmode{[}DERasterBand{]}
Raster grid used to align output DEM with other related grids or adjacent local folders.

\item[{\sphinxstylestrong{NHD Waterbody Grid}}] \leavevmode{[}DERasterDataset (optional){]}
Grid representing waterbodies from the bathymetric gradient step.

\item[{\sphinxstylestrong{NHD Flowline Grid}}] \leavevmode{[}DERasterDataset (optional){]}
Grid representing flowlines from the bathymetric gradient step.

\item[{\sphinxstylestrong{Inner Walls}}] \leavevmode{[}DEFeatureClass (optional){]}
Polyline feature class used to inforce internal drainage.

\item[{\sphinxstylestrong{Cell Size}}] \leavevmode{[}GPString{]}
Output cell size, defaults to 10.

\item[{\sphinxstylestrong{Drain Plugs}}] \leavevmode{[}DEFeatureClass (optional){]}
\item[{\sphinxstylestrong{HUC buffer}}] \leavevmode{[}GPDouble (optional){]}
Distance to buffer the HUC layer, dfaults to 50.

\item[{\sphinxstylestrong{Inner Wall Buffer}}] \leavevmode{[}GPDouble (optional){]}
Distance to buffer the inwall, defaults to 15.

\item[{\sphinxstylestrong{Inner Wall Height}}] \leavevmode{[}GPDouble (optional){]}
Inwall height, defaults to 150000.

\item[{\sphinxstylestrong{Outer Wall Height}}] \leavevmode{[}GPDouble (optional){]}
Outer wall height, defaults to 300000.

\item[{\sphinxstylestrong{AGREE buffer}}] \leavevmode{[}GPDouble (optional){]}
Defaults to 60.

\item[{\sphinxstylestrong{AGREE Smooth Drop}}] \leavevmode{[}GPDouble (optional){]}
Defaults to -500.

\item[{\sphinxstylestrong{AGREE Sharp Drop}}] \leavevmode{[}GPDouble (optional){]}
Defaults to -50000.

\item[{\sphinxstylestrong{Bowl Depth}}] \leavevmode{[}GPDouble (optional){]}
Defaults to 2000.

\end{description}

\item[{Returns}] \leavevmode\begin{description}
\item[{\sphinxstylestrong{parameters}}] \leavevmode{[}list{]}
List of input parameters passed to the execute method.

\end{description}

\end{description}\end{quote}

\end{fulllineitems}


\end{fulllineitems}



\subsection{Adjust Accumulation}
\label{\detokenize{StreamStats_DataPrep:adjust-accumulation}}\index{AdjustAccum (class in StreamStats\_DataPrep)@\spxentry{AdjustAccum}\spxextra{class in StreamStats\_DataPrep}}

\begin{fulllineitems}
\phantomsection\label{\detokenize{StreamStats_DataPrep:StreamStats_DataPrep.AdjustAccum}}\pysigline{\sphinxbfcode{\sphinxupquote{class }}\sphinxcode{\sphinxupquote{StreamStats\_DataPrep.}}\sphinxbfcode{\sphinxupquote{AdjustAccum}}}
Adjust flow accumulation grids following hydro-enforcement.

This tool is a wrapper on {\hyperref[\detokenize{make_hydrodem:make_hydrodem.adjust_accum}]{\sphinxcrossref{\sphinxcode{\sphinxupquote{make\_hydrodem.adjust\_accum()}}}}}.
\subsubsection*{Methods}


\begin{savenotes}\sphinxatlongtablestart\begin{longtable}{\X{1}{2}\X{1}{2}}
\hline

\endfirsthead

\multicolumn{2}{c}%
{\makebox[0pt]{\sphinxtablecontinued{\tablename\ \thetable{} -- continued from previous page}}}\\
\hline

\endhead

\hline
\multicolumn{2}{r}{\makebox[0pt][r]{\sphinxtablecontinued{Continued on next page}}}\\
\endfoot

\endlastfoot

{\hyperref[\detokenize{StreamStats_DataPrep:StreamStats_DataPrep.AdjustAccum.getParameterInfo}]{\sphinxcrossref{\sphinxcode{\sphinxupquote{getParameterInfo}}}}}(self)
&
Adjust flow accumulation grid inputs.
\\
\hline
\end{longtable}\sphinxatlongtableend\end{savenotes}
\index{getParameterInfo() (StreamStats\_DataPrep.AdjustAccum method)@\spxentry{getParameterInfo()}\spxextra{StreamStats\_DataPrep.AdjustAccum method}}

\begin{fulllineitems}
\phantomsection\label{\detokenize{StreamStats_DataPrep:StreamStats_DataPrep.AdjustAccum.getParameterInfo}}\pysiglinewithargsret{\sphinxbfcode{\sphinxupquote{getParameterInfo}}}{\emph{self}}{}
Adjust flow accumulation grid inputs.
\begin{quote}\begin{description}
\item[{Parameters}] \leavevmode\begin{description}
\item[{\sphinxstylestrong{Downstream Accumulation Grid}}] \leavevmode{[}DERasterDataset{]}
Downstream raster dataset to correct.

\item[{\sphinxstylestrong{Downstream Flow Direction Grid}}] \leavevmode{[}DERasterDataset{]}
Downstream flow direction grid.

\item[{\sphinxstylestrong{Upstream Flow Accumulation Grids}}] \leavevmode{[}DERasterDataset{]}
Upstream flow accumulation grids to correct the downstream grid with.

\item[{\sphinxstylestrong{Upstream Flow Direction Grids}}] \leavevmode{[}DERasterDataset{]}
Upstream flow direction grids corresponding to the grids listed above.

\item[{\sphinxstylestrong{Workspace}}] \leavevmode{[}Workspace (Geodatabase){]}
Geodatabase to work in.

\end{description}

\item[{Returns}] \leavevmode\begin{description}
\item[{\sphinxstylestrong{parameters}}] \leavevmode{[}list{]}
List of input parameters passed to the execute method.

\end{description}

\end{description}\end{quote}

\end{fulllineitems}


\end{fulllineitems}



\subsection{Adjust Accumulation Simple}
\label{\detokenize{StreamStats_DataPrep:adjust-accumulation-simple}}\index{AdjustAccumSimp (class in StreamStats\_DataPrep)@\spxentry{AdjustAccumSimp}\spxextra{class in StreamStats\_DataPrep}}

\begin{fulllineitems}
\phantomsection\label{\detokenize{StreamStats_DataPrep:StreamStats_DataPrep.AdjustAccumSimp}}\pysigline{\sphinxbfcode{\sphinxupquote{class }}\sphinxcode{\sphinxupquote{StreamStats\_DataPrep.}}\sphinxbfcode{\sphinxupquote{AdjustAccumSimp}}}
Simply adjust a flow accumulation grid.

This tool is a wrapper on {\hyperref[\detokenize{make_hydrodem:make_hydrodem.adjust_accum_simple}]{\sphinxcrossref{\sphinxcode{\sphinxupquote{make\_hydrodem.adjust\_accum\_simple()}}}}}.
\subsubsection*{Methods}


\begin{savenotes}\sphinxatlongtablestart\begin{longtable}{\X{1}{2}\X{1}{2}}
\hline

\endfirsthead

\multicolumn{2}{c}%
{\makebox[0pt]{\sphinxtablecontinued{\tablename\ \thetable{} -- continued from previous page}}}\\
\hline

\endhead

\hline
\multicolumn{2}{r}{\makebox[0pt][r]{\sphinxtablecontinued{Continued on next page}}}\\
\endfoot

\endlastfoot

{\hyperref[\detokenize{StreamStats_DataPrep:StreamStats_DataPrep.AdjustAccumSimp.getParameterInfo}]{\sphinxcrossref{\sphinxcode{\sphinxupquote{getParameterInfo}}}}}(self)
&
Simple flow accumulation grid adjustment inputs.
\\
\hline
\end{longtable}\sphinxatlongtableend\end{savenotes}
\index{getParameterInfo() (StreamStats\_DataPrep.AdjustAccumSimp method)@\spxentry{getParameterInfo()}\spxextra{StreamStats\_DataPrep.AdjustAccumSimp method}}

\begin{fulllineitems}
\phantomsection\label{\detokenize{StreamStats_DataPrep:StreamStats_DataPrep.AdjustAccumSimp.getParameterInfo}}\pysiglinewithargsret{\sphinxbfcode{\sphinxupquote{getParameterInfo}}}{\emph{self}}{}
Simple flow accumulation grid adjustment inputs.
\begin{quote}\begin{description}
\item[{Parameters}] \leavevmode\begin{description}
\item[{\sphinxstylestrong{Inlet Point}}] \leavevmode{[}GPFeatureLayer{]}
Point feature class indicating the inlet to the downstream hydrologic unit.

\item[{\sphinxstylestrong{Flow Direction Grid}}] \leavevmode{[}DERasterBand{]}
Flow direction grid of the downstream hydrologic unit.

\item[{\sphinxstylestrong{Flow Accumulation Grid}}] \leavevmode{[}DERasterBand{]}
Flow accumuation grid of the downstream hydrologic unit.

\item[{\sphinxstylestrong{HydroDEM}}] \leavevmode{[}DERasterBand{]}
Downstream hydrologic unit hydro-enforced digital elevation model.

\item[{\sphinxstylestrong{Output FAC}}] \leavevmode{[}DERasterBand{]}
Corrected flow accumuation grid, defaults to hydrodemfac\_global.

\item[{\sphinxstylestrong{Adjustment Value}}] \leavevmode{[}GPString{]}
Upstream flow accumulation value to correct the downstream hydrologic unit with, defaults to 150000 grid cells.

\end{description}

\item[{Returns}] \leavevmode\begin{description}
\item[{\sphinxstylestrong{parameters}}] \leavevmode{[}list{]}
List of input parameters passed to the execute method.

\end{description}

\end{description}\end{quote}

\end{fulllineitems}


\end{fulllineitems}



\subsection{Post Hydrodem}
\label{\detokenize{StreamStats_DataPrep:post-hydrodem}}\index{posthydrodem (class in StreamStats\_DataPrep)@\spxentry{posthydrodem}\spxextra{class in StreamStats\_DataPrep}}

\begin{fulllineitems}
\phantomsection\label{\detokenize{StreamStats_DataPrep:StreamStats_DataPrep.posthydrodem}}\pysigline{\sphinxbfcode{\sphinxupquote{class }}\sphinxcode{\sphinxupquote{StreamStats\_DataPrep.}}\sphinxbfcode{\sphinxupquote{posthydrodem}}}
ArcHydro processing using the hydro-enforced digital elevation model and resultant flow direction and flow accumulation grids.

This tool is a wrapper on {\hyperref[\detokenize{make_hydrodem:make_hydrodem.postHydroDEM}]{\sphinxcrossref{\sphinxcode{\sphinxupquote{make\_hydrodem.postHydroDEM()}}}}}.
\subsubsection*{Notes}

This tool only functions with ArcMap / Python 2, ArcPro / Python 3 are currently not supported.
\subsubsection*{Methods}


\begin{savenotes}\sphinxatlongtablestart\begin{longtable}{\X{1}{2}\X{1}{2}}
\hline

\endfirsthead

\multicolumn{2}{c}%
{\makebox[0pt]{\sphinxtablecontinued{\tablename\ \thetable{} -- continued from previous page}}}\\
\hline

\endhead

\hline
\multicolumn{2}{r}{\makebox[0pt][r]{\sphinxtablecontinued{Continued on next page}}}\\
\endfoot

\endlastfoot

{\hyperref[\detokenize{StreamStats_DataPrep:StreamStats_DataPrep.posthydrodem.getParameterInfo}]{\sphinxcrossref{\sphinxcode{\sphinxupquote{getParameterInfo}}}}}(self)
&
Post hydro-enforcement processing inputs.
\\
\hline
\end{longtable}\sphinxatlongtableend\end{savenotes}
\index{getParameterInfo() (StreamStats\_DataPrep.posthydrodem method)@\spxentry{getParameterInfo()}\spxextra{StreamStats\_DataPrep.posthydrodem method}}

\begin{fulllineitems}
\phantomsection\label{\detokenize{StreamStats_DataPrep:StreamStats_DataPrep.posthydrodem.getParameterInfo}}\pysiglinewithargsret{\sphinxbfcode{\sphinxupquote{getParameterInfo}}}{\emph{self}}{}
Post hydro-enforcement processing inputs.
\begin{quote}\begin{description}
\item[{Parameters}] \leavevmode\begin{description}
\item[{\sphinxstylestrong{Workspace}}] \leavevmode{[}DEWorkspace (Geodatabase){]}
Geodatabase to work in.

\item[{\sphinxstylestrong{hydrodemfac}}] \leavevmode{[}DERasterDataset{]}
Hydro-enforced flow accumulation grid.

\item[{\sphinxstylestrong{hydrodemfdr}}] \leavevmode{[}DERasterDataset{]}
Hydro-enforced flow direction grid.

\item[{\sphinxstylestrong{str threshold}}] \leavevmode{[}GPLong{]}
Stream threshold in raster cells.

\item[{\sphinxstylestrong{str900 threshold}}] \leavevmode{[}GPLong{]}
str900 grid threshold, in raster cells.

\item[{\sphinxstylestrong{Sink Link}}] \leavevmode{[}DERasterBand{]}
Sink link raster name.

\end{description}

\item[{Returns}] \leavevmode\begin{description}
\item[{\sphinxstylestrong{parameters}}] \leavevmode{[}list{]}
List of input parameters passed to the execute method.

\end{description}

\end{description}\end{quote}

\end{fulllineitems}


\end{fulllineitems}



\section{StreamStats DataPrep Library}
\label{\detokenize{modules:streamstats-dataprep-library}}\label{\detokenize{modules::doc}}

\subsection{databaseSetup Module}
\label{\detokenize{databaseSetup:module-databaseSetup}}\label{\detokenize{databaseSetup:databasesetup-module}}\label{\detokenize{databaseSetup::doc}}\index{databaseSetup (module)@\spxentry{databaseSetup}\spxextra{module}}
This library creates the folder structure and does not data management to facilitate preparing data for use in StreamStats.
\index{databaseSetup() (in module databaseSetup)@\spxentry{databaseSetup()}\spxextra{in module databaseSetup}}

\begin{fulllineitems}
\phantomsection\label{\detokenize{databaseSetup:databaseSetup.databaseSetup}}\pysiglinewithargsret{\sphinxcode{\sphinxupquote{databaseSetup.}}\sphinxbfcode{\sphinxupquote{databaseSetup}}}{\emph{output\_workspace}, \emph{output\_gdb\_name}, \emph{hu\_dataset}, \emph{hu8\_field}, \emph{hu12\_field}, \emph{hucbuffer}, \emph{nhd\_path}, \emph{elevation\_projection\_template}, \emph{alt\_buff}, \emph{version=None}}{}
Set up the local folders and copy hydrography data.

This tool creates folder cooresponding to each local hydrologic unit and fills those folders with the flowlines, inwalls, and outwalls that will be used later to hydro-enforce the digital elevation model for each hydrologic unit. This tool also creates a global geodatabase with a feature class for the whole domain.
\begin{quote}\begin{description}
\item[{Parameters}] \leavevmode\begin{description}
\item[{\sphinxstylestrong{output\_workspace}}] \leavevmode{[}str{]}
Output directory for processing to occure in.

\item[{\sphinxstylestrong{output\_gdb\_name}}] \leavevmode{[}str{]}
Global file geodatabase to be created.

\item[{\sphinxstylestrong{hu\_dataset}}] \leavevmode{[}str{]}
Feature class that defines local folder geographic boundaries.

\item[{\sphinxstylestrong{hu8\_field}}] \leavevmode{[}str{]}
Field name in “hu\_dataset” to dissolve boundaries to local folder extents.

\item[{\sphinxstylestrong{hu12\_field}}] \leavevmode{[}str{]}
Field name in “hu\_dataset” to generate inwalls from.

\item[{\sphinxstylestrong{hucbuffer}}] \leavevmode{[}str{]}
Distance to buffer local folder bounds in map units.

\item[{\sphinxstylestrong{nhd\_path}}] \leavevmode{[}str{]}
Path to workspace containing NHD geodatabases.

\item[{\sphinxstylestrong{elevation\_projection\_template}}] \leavevmode{[}str{]}
Path to DEM file to use as a projection template.

\item[{\sphinxstylestrong{alt\_buff}}] \leavevmode{[}str{]}
Alternative buffer to use on local folder boundaries.

\item[{\sphinxstylestrong{version}}] \leavevmode{[}str{]}
Package version number.

\end{description}

\item[{Returns}] \leavevmode\begin{description}
\item[{None}] \leavevmode
\end{description}

\end{description}\end{quote}
\subsubsection*{Notes}

As this tool moves through each local hydrologic unit it searches the \sphinxstyleemphasis{nhd\_path} for a geodatabase with hydrography data with the same HUC-4 as the local hydrologic unit. If this cannot be found the tool will skip that local hydrologic unit. Non-NHD hydrography data can be used with this tool, but it must be named and organized in the same way that NHD hydrography is.

\end{fulllineitems}



\subsection{elevationTools Module}
\label{\detokenize{elevationTools:module-elevationTools}}\label{\detokenize{elevationTools:elevationtools-module}}\label{\detokenize{elevationTools::doc}}\index{elevationTools (module)@\spxentry{elevationTools}\spxextra{module}}
This library is a collection of functions to create an seamless mosaic of digital elevation models for an area of interest, extracts a hydrologic unit (local folder) from this mosaic, checks for and fills no data values in the digital elevation model, projects the digital elevation model to the target projection, and scales the elevation values for better data storage.
\index{checkNoData() (in module elevationTools)@\spxentry{checkNoData()}\spxextra{in module elevationTools}}

\begin{fulllineitems}
\phantomsection\label{\detokenize{elevationTools:elevationTools.checkNoData}}\pysiglinewithargsret{\sphinxcode{\sphinxupquote{elevationTools.}}\sphinxbfcode{\sphinxupquote{checkNoData}}}{\emph{InGrid}, \emph{tmpLoc}, \emph{OutPolys\_shp}, \emph{version=None}}{}
Generates a feature class of no data values.
\begin{quote}\begin{description}
\item[{Parameters}] \leavevmode\begin{description}
\item[{\sphinxstylestrong{InGrid}}] \leavevmode{[}Raster{]}
Input DEM grid to search for no data values

\item[{\sphinxstylestrong{tmpLoc}}] \leavevmode{[}str{]}
Path to workspace

\item[{\sphinxstylestrong{OutPoly\_shp}}] \leavevmode{[}str{]}
Name for output feature class

\item[{\sphinxstylestrong{version}}] \leavevmode{[}str, optional{]}
StreamStats DataPrepTools version

\end{description}

\item[{Returns}] \leavevmode\begin{description}
\item[{\sphinxstylestrong{featCount}}] \leavevmode{[}int{]}
Count of no data features generated.

\end{description}

\end{description}\end{quote}

\end{fulllineitems}

\index{compareSpatialRefUnits() (in module elevationTools)@\spxentry{compareSpatialRefUnits()}\spxextra{in module elevationTools}}

\begin{fulllineitems}
\phantomsection\label{\detokenize{elevationTools:elevationTools.compareSpatialRefUnits}}\pysiglinewithargsret{\sphinxcode{\sphinxupquote{elevationTools.}}\sphinxbfcode{\sphinxupquote{compareSpatialRefUnits}}}{\emph{grd}}{}
Compare horizontal and vertical units from a raster dataset. Returns True if units are the same, returns False if they are different.
\begin{quote}\begin{description}
\item[{Parameters}] \leavevmode\begin{description}
\item[{\sphinxstylestrong{grd}}] \leavevmode{[}str{]}
Path to raster dataset.

\end{description}

\item[{Returns}] \leavevmode\begin{description}
\item[{\sphinxstylestrong{sameUnits}}] \leavevmode{[}bool{]}
True if units are the same, False if not.

\end{description}

\end{description}\end{quote}

\end{fulllineitems}

\index{elevIndex() (in module elevationTools)@\spxentry{elevIndex()}\spxextra{in module elevationTools}}

\begin{fulllineitems}
\phantomsection\label{\detokenize{elevationTools:elevationTools.elevIndex}}\pysiglinewithargsret{\sphinxcode{\sphinxupquote{elevationTools.}}\sphinxbfcode{\sphinxupquote{elevIndex}}}{\emph{OutLoc}, \emph{rcName}, \emph{coordsysRaster}, \emph{InputELEVDATAws}, \emph{version=None}}{}
Make a raster mosaic dataset of the digital elevation models found in \sphinxstyleemphasis{InputELEVDATAws}.
\begin{quote}\begin{description}
\item[{Parameters}] \leavevmode\begin{description}
\item[{\sphinxstylestrong{OutLoc}}] \leavevmode{[}str{]}
Path to output location for the raster catalog.

\item[{\sphinxstylestrong{rcName}}] \leavevmode{[}str{]}
Name of the output mosaic raster dataset.

\item[{\sphinxstylestrong{coordsysRaster}}] \leavevmode{[}str{]}
Path to raster from which to base the mosaic dataset’s coordinate system.

\item[{\sphinxstylestrong{InputELEVDATAws}}] \leavevmode{[}str{]}
Path to workspace containing the elevation data to be included in the mosaic raster dataset. Rasters in this workspace should be either geoTiffs or ESRI grids. Rasters must be unzipped, but they can be in subfolders.

\item[{\sphinxstylestrong{version}}] \leavevmode{[}str, optional{]}
StreamStats DataPrepTools version number.

\end{description}

\item[{Returns}] \leavevmode\begin{description}
\item[{None}] \leavevmode
\end{description}

\end{description}\end{quote}

\end{fulllineitems}

\index{extractPoly() (in module elevationTools)@\spxentry{extractPoly()}\spxextra{in module elevationTools}}

\begin{fulllineitems}
\phantomsection\label{\detokenize{elevationTools:elevationTools.extractPoly}}\pysiglinewithargsret{\sphinxcode{\sphinxupquote{elevationTools.}}\sphinxbfcode{\sphinxupquote{extractPoly}}}{\emph{Input\_Workspace}, \emph{nedindx}, \emph{clpfeat}, \emph{OutGrd}, \emph{version=None}}{}
Extracts watershed DEM from a raster catalogue of tiles based on a watershed feature.
\begin{quote}\begin{description}
\item[{Parameters}] \leavevmode\begin{description}
\item[{\sphinxstylestrong{Input\_Workspace}}] \leavevmode{[}str{]}
Path to the workspace to work in.

\item[{\sphinxstylestrong{nedindx}}] \leavevmode{[}str{]}
Path to the elevation data mosaic dataset.

\item[{\sphinxstylestrong{clpfeat}}] \leavevmode{[}str{]}
Path to the clipping feature.

\item[{\sphinxstylestrong{OutGrd}}] \leavevmode{[}str{]}
Name of the output grid to be generated in Input\_Workspace.

\item[{\sphinxstylestrong{version}}] \leavevmode{[}str, optional{]}
StreamStats DataPrepTools version number.

\end{description}

\item[{Returns}] \leavevmode\begin{description}
\item[{None}] \leavevmode
\end{description}

\end{description}\end{quote}

\end{fulllineitems}

\index{fillNoData() (in module elevationTools)@\spxentry{fillNoData()}\spxextra{in module elevationTools}}

\begin{fulllineitems}
\phantomsection\label{\detokenize{elevationTools:elevationTools.fillNoData}}\pysiglinewithargsret{\sphinxcode{\sphinxupquote{elevationTools.}}\sphinxbfcode{\sphinxupquote{fillNoData}}}{\emph{workspace}, \emph{InGrid}, \emph{OutGrid}, \emph{version=None}}{}
Replaces NODATA values in a grid with mean values within 3x3 window.
\begin{quote}\begin{description}
\item[{Parameters}] \leavevmode\begin{description}
\item[{\sphinxstylestrong{workspace}}] \leavevmode{[}str{]}
Path to the workspace to work in.

\item[{\sphinxstylestrong{InGrid}}] \leavevmode{[}str{]}
Name of the input grid to be filled.

\item[{\sphinxstylestrong{OutGrid}}] \leavevmode{[}str{]}
Name of the output grid.

\item[{\sphinxstylestrong{Version}}] \leavevmode{[}str, optional{]}
Code version

\end{description}

\item[{Returns}] \leavevmode\begin{description}
\item[{None}] \leavevmode
\end{description}

\end{description}\end{quote}
\subsubsection*{Notes}

May be run repeatedly to fill in areas wider than 2 cells. Note the output is floating point, even if the input is integer. Note this will expand the data area of the grid around the outer edges of data, in addition to filling in NODATA gaps in the interior of the grid.

Converted from model builder to arcpy, Theodore Barnhart, \sphinxhref{mailto:tbarnhart@usgs.gov}{tbarnhart@usgs.gov}, 20190222

\end{fulllineitems}

\index{projScale() (in module elevationTools)@\spxentry{projScale()}\spxextra{in module elevationTools}}

\begin{fulllineitems}
\phantomsection\label{\detokenize{elevationTools:elevationTools.projScale}}\pysiglinewithargsret{\sphinxcode{\sphinxupquote{elevationTools.}}\sphinxbfcode{\sphinxupquote{projScale}}}{\emph{Input\_Workspace}, \emph{InGrd}, \emph{OutGrd}, \emph{OutCoordsys}, \emph{OutCellSize}, \emph{RegistrationPoint}, \emph{scaleFact=100}, \emph{version=None}}{}~\begin{description}
\item[{Projects a NED grid to a user-specified coordinate system, handling cell registration. Converts}] \leavevmode
output grid to centimeters (multiplies by 100 and rounds).

\end{description}
\begin{quote}\begin{description}
\item[{Parameters}] \leavevmode\begin{description}
\item[{\sphinxstylestrong{Input\_Workspace}}] \leavevmode{[}str{]}
Path to input workspace.

\item[{\sphinxstylestrong{InGrd}}] \leavevmode{[}str{]}
Name of the grid to be projected and scaled.

\item[{\sphinxstylestrong{OutGrd}}] \leavevmode{[}str{]}
Name of the output grid.

\item[{\sphinxstylestrong{OutCoordsys}}] \leavevmode{[}str{]}
Path to the dataset to base the projection off of.

\item[{\sphinxstylestrong{OutCellSize}}] \leavevmode{[}int or float{]}
Cell size for output grid.

\item[{\sphinxstylestrong{RegistrationPoint}}] \leavevmode{[}str{]}
Registration point for output grid so all grids snap correctly. In the format “0 0” where the zeros are the x and y of the registration point.

\item[{\sphinxstylestrong{version}}] \leavevmode{[}str{]}
Stream Stats version number.

\end{description}

\item[{Returns}] \leavevmode\begin{description}
\item[{None}] \leavevmode
\end{description}

\end{description}\end{quote}

\end{fulllineitems}



\subsection{make\_hydrodem Module}
\label{\detokenize{make_hydrodem:module-make_hydrodem}}\label{\detokenize{make_hydrodem:make-hydrodem-module}}\label{\detokenize{make_hydrodem::doc}}\index{make\_hydrodem (module)@\spxentry{make\_hydrodem}\spxextra{module}}
Code to replicate the hydroDEM\_work\_mod.aml, agree.aml, and fill.aml scripts

Theodore Barnhart, \sphinxhref{mailto:tbarnhart@usgs.gov}{tbarnhart@usgs.gov}, 20190225

Reference: agree.aml
\index{SnapExtent() (in module make\_hydrodem)@\spxentry{SnapExtent()}\spxextra{in module make\_hydrodem}}

\begin{fulllineitems}
\phantomsection\label{\detokenize{make_hydrodem:make_hydrodem.SnapExtent}}\pysiglinewithargsret{\sphinxcode{\sphinxupquote{make\_hydrodem.}}\sphinxbfcode{\sphinxupquote{SnapExtent}}}{\emph{lExtent}, \emph{lRaster}}{}
Returns a given extent snapped to the passed raster.
\begin{quote}\begin{description}
\item[{Parameters}] \leavevmode\begin{description}
\item[{\sphinxstylestrong{lExtent}}] \leavevmode{[}str{]}
ESRI Arcpy extent string

\item[{\sphinxstylestrong{lRaster}}] \leavevmode{[}str{]}
Path to raster dataset

\end{description}

\item[{Returns}] \leavevmode\begin{description}
\item[{\sphinxstylestrong{extent}}] \leavevmode{[}str{]}
ESRI ArcPy extent string

\end{description}

\end{description}\end{quote}

\end{fulllineitems}

\index{adjust\_accum() (in module make\_hydrodem)@\spxentry{adjust\_accum()}\spxextra{in module make\_hydrodem}}

\begin{fulllineitems}
\phantomsection\label{\detokenize{make_hydrodem:make_hydrodem.adjust_accum}}\pysiglinewithargsret{\sphinxcode{\sphinxupquote{make\_hydrodem.}}\sphinxbfcode{\sphinxupquote{adjust\_accum}}}{\emph{facPth}, \emph{fdrPth}, \emph{upstreamFACpths}, \emph{upstreamFDRpths}, \emph{workspace}, \emph{version=None}}{}~\begin{quote}\begin{description}
\item[{Parameters}] \leavevmode\begin{description}
\item[{\sphinxstylestrong{facPth}}] \leavevmode{[}str{]}
Path to downstream flow accumulation grid

\item[{\sphinxstylestrong{fdrPth}}] \leavevmode{[}str{]}
Path to downstream flow direction grid

\item[{\sphinxstylestrong{upstreamFACpths}}] \leavevmode{[}list{]}
List of paths to upstream flow accumulation grids

\item[{\sphinxstylestrong{upstreamFDRpths}}] \leavevmode{[}list{]}
List of paths to upstream flow direction grids

\item[{\sphinxstylestrong{workspace}}] \leavevmode{[}str{]}
local geodatabase to work in.

\item[{\sphinxstylestrong{version}}] \leavevmode{[}str (optional){]}
Stream Stats datapreptool version number.

\end{description}

\end{description}\end{quote}

\end{fulllineitems}

\index{adjust\_accum\_simple() (in module make\_hydrodem)@\spxentry{adjust\_accum\_simple()}\spxextra{in module make\_hydrodem}}

\begin{fulllineitems}
\phantomsection\label{\detokenize{make_hydrodem:make_hydrodem.adjust_accum_simple}}\pysiglinewithargsret{\sphinxcode{\sphinxupquote{make\_hydrodem.}}\sphinxbfcode{\sphinxupquote{adjust\_accum\_simple}}}{\emph{ptin}, \emph{fdrin}, \emph{facin}, \emph{filin}, \emph{facout}, \emph{incrval}, \emph{version=None}}{}
Simple drainage adjust method.

Original coding by Al Rea (2010) \sphinxhref{mailto:ahrea@usgs.gov}{ahrea@usgs.gov}
Updated to arcPy by Theodore Barnhart (2019) \sphinxhref{mailto:tbarnhart@usgs.gov}{tbarnhart@usgs.gov}

Adds a value to the flow accumulation grid given an input point using a least-cost-path to coascalde down through the flow direction grid.
\begin{quote}\begin{description}
\item[{Parameters}] \leavevmode\begin{description}
\item[{\sphinxstylestrong{ptin}}] \leavevmode{[}str (feature class){]}
Point feature class representing one inlet to the downstream DEM.

\item[{\sphinxstylestrong{fdrin}}] \leavevmode{[}str (raster){]}
Flow direction raster

\item[{\sphinxstylestrong{facin}}] \leavevmode{[}str (raster){]}
Name of the flow accumulation raster

\item[{\sphinxstylestrong{filin}}] \leavevmode{[}str (raster){]}
Burned DEM to use as cost surface.

\item[{\sphinxstylestrong{facout}}] \leavevmode{[}str (raster){]}
Output name of adjusted FAC grid.

\item[{\sphinxstylestrong{incrval}}] \leavevmode{[}int{]}
Value to adjust the downstream FAC grid by.

\item[{\sphinxstylestrong{version}}] \leavevmode{[}str{]}
Stream Stats version number

\end{description}

\item[{Returns}] \leavevmode\begin{description}
\item[{None}] \leavevmode
\end{description}

\end{description}\end{quote}

\end{fulllineitems}

\index{agree() (in module make\_hydrodem)@\spxentry{agree()}\spxextra{in module make\_hydrodem}}

\begin{fulllineitems}
\phantomsection\label{\detokenize{make_hydrodem:make_hydrodem.agree}}\pysiglinewithargsret{\sphinxcode{\sphinxupquote{make\_hydrodem.}}\sphinxbfcode{\sphinxupquote{agree}}}{\emph{origdem}, \emph{dendrite}, \emph{agreebuf}, \emph{agreesmooth}, \emph{agreesharp}}{}
Agree function from AGREE.aml

Original function by Ferdi Hellweger, \sphinxurl{http://www.ce.utexas.edu/prof/maidment/gishydro/ferdi/research/agree/agree.html}

recoded by Theodore Barnhart, \sphinxhref{mailto:tbarnhart@usgs.gov}{tbarnhart@usgs.gov}, 20190225

— Creation Information —

Name: agree.aml
Version: 1.1
Date: 10/13/96
Author: Ferdi Hellweger
\begin{quote}

Center for Research in Water Resources
The University of Texas at Austin
\sphinxhref{mailto:ferdi@crwr.utexas.edu}{ferdi@crwr.utexas.edu}
\end{quote}

— Purpose/Description —

AGREE is a surface reconditioning system for Digital Elevation Models (DEMs).
The system adjusts the surface elevation of the DEM to be consistent with a
vector coverage.  The vecor coverage can be a stream or ridge line coverage.
\begin{quote}\begin{description}
\item[{Parameters}] \leavevmode\begin{description}
\item[{\sphinxstylestrong{origdem}}] \leavevmode{[}arcpy.sa Raster{]}
Original DEM with the desired cell size, oelevgrid in original script

\item[{\sphinxstylestrong{dendrite}}] \leavevmode{[}arcpy.sa Raster{]}
Dendrite feature layer to adjust the DEM, vectcov in the original script

\item[{\sphinxstylestrong{agreebuf}}] \leavevmode{[}float{]}
Buffer smoothing distance (same units as the horizontal), buffer in original script

\item[{\sphinxstylestrong{agreesmooth}}] \leavevmode{[}float{]}
Smoothing distance (same units as the vertical), smoothdist in the original script

\item[{\sphinxstylestrong{agreesharp}}] \leavevmode{[}float{]}
Distance for sharp feature (same units as the vertical), sharpdist in the original script

\end{description}

\item[{Returns}] \leavevmode\begin{description}
\item[{\sphinxstylestrong{elevgrid}}] \leavevmode{[}arcpy.sa Raster{]}
conditioned elevation grid returned as a arcpy.sa Raster object

\end{description}

\end{description}\end{quote}

\end{fulllineitems}

\index{bathymetricGradient() (in module make\_hydrodem)@\spxentry{bathymetricGradient()}\spxextra{in module make\_hydrodem}}

\begin{fulllineitems}
\phantomsection\label{\detokenize{make_hydrodem:make_hydrodem.bathymetricGradient}}\pysiglinewithargsret{\sphinxcode{\sphinxupquote{make\_hydrodem.}}\sphinxbfcode{\sphinxupquote{bathymetricGradient}}}{\emph{workspace}, \emph{snapGrid}, \emph{hucPoly}, \emph{hydrographyArea}, \emph{hydrographyFlowline}, \emph{hydrographyWaterbody}, \emph{cellsize}, \emph{version=None}}{}
Generates the input datasets for enforcing a bathymetic gradient in hydroDEM (bowling).
\begin{description}
\item[{Originally:}] \leavevmode
ssbowling.py
Created on: Wed Jan 31 2007 01:16:48 PM
Author:  Martyn Smith
USGS New York Water Science Center Troy, NY

\end{description}

Updated to Arcpy, 20190222, Theodore Barnhart, \sphinxhref{mailto:tbarnhart@usgs.gov}{tbarnhart@usgs.gov}

This script takes a set of NHD Hydrography Datasets, extracts the appropriate
features and converts them to rasters for the Bathymetric Gradient (bowling) inputs to HydroDEM
\begin{quote}\begin{description}
\item[{Parameters}] \leavevmode\begin{description}
\item[{\sphinxstylestrong{workspace}}] \leavevmode{[}str{]}
\item[{\sphinxstylestrong{snapGrid}}] \leavevmode{[}str{]}
\item[{\sphinxstylestrong{hucPoly}}] \leavevmode{[}str{]}
\item[{\sphinxstylestrong{hydrographyArea}}] \leavevmode{[}str{]}
\item[{\sphinxstylestrong{hydrographyFlowline}}] \leavevmode{[}str{]}
\item[{\sphinxstylestrong{hydrographyWaterbody}}] \leavevmode{[}str{]}
\item[{\sphinxstylestrong{cellsize}}] \leavevmode{[}str{]}
\item[{\sphinxstylestrong{version}}] \leavevmode{[}str{]}
Package version number

\end{description}

\end{description}\end{quote}

\end{fulllineitems}

\index{coastaldem() (in module make\_hydrodem)@\spxentry{coastaldem()}\spxextra{in module make\_hydrodem}}

\begin{fulllineitems}
\phantomsection\label{\detokenize{make_hydrodem:make_hydrodem.coastaldem}}\pysiglinewithargsret{\sphinxcode{\sphinxupquote{make\_hydrodem.}}\sphinxbfcode{\sphinxupquote{coastaldem}}}{\emph{Input\_Workspace}, \emph{grdNamePth}, \emph{InFeatureClass}, \emph{OutRaster}, \emph{seaLevel}}{}
Sets elevations for water and other areas in DEM
\begin{description}
\item[{Originally:}] \leavevmode
Al Rea, \sphinxhref{mailto:ahrea@usgs.gov}{ahrea@usgs.gov}, 05/01/2010, original coding
ahrea, 10/30/2010 updated with more detailed comments
Theodore Barnhart, 20190225, \sphinxhref{mailto:tbarnhart@usgs.gov}{tbarnhart@usgs.gov}, updated to arcpy

\end{description}
\begin{quote}\begin{description}
\item[{Parameters}] \leavevmode\begin{description}
\item[{\sphinxstylestrong{Input\_Workspace}}] \leavevmode{[}str{]}
Input workspace, output raster will be written here.

\item[{\sphinxstylestrong{grdNamePth}}] \leavevmode{[}str{]}
Input DEM grid.

\item[{\sphinxstylestrong{InFeatureClass}}] \leavevmode{[}str{]}
LandSea feature class.

\item[{\sphinxstylestrong{OutRaster}}] \leavevmode{[}str{]}
Output DEM grid name.

\item[{\sphinxstylestrong{seaLevel}}] \leavevmode{[}float{]}
Elevation at which to make the sea

\end{description}

\item[{Returns}] \leavevmode\begin{description}
\item[{\sphinxstylestrong{OutRaster}}] \leavevmode{[}raster{]}
Output raster written to the workspace.

\end{description}

\end{description}\end{quote}

\end{fulllineitems}

\index{hydrodem() (in module make\_hydrodem)@\spxentry{hydrodem()}\spxextra{in module make\_hydrodem}}

\begin{fulllineitems}
\phantomsection\label{\detokenize{make_hydrodem:make_hydrodem.hydrodem}}\pysiglinewithargsret{\sphinxcode{\sphinxupquote{make\_hydrodem.}}\sphinxbfcode{\sphinxupquote{hydrodem}}}{\emph{outdir}, \emph{huc8cov}, \emph{origdemPth}, \emph{dendrite}, \emph{snap\_grid}, \emph{bowl\_polys}, \emph{bowl\_lines}, \emph{inwall}, \emph{drainplug}, \emph{buffdist}, \emph{inwallbuffdist}, \emph{inwallht}, \emph{outwallht}, \emph{agreebuf}, \emph{agreesmooth}, \emph{agreesharp}, \emph{bowldepth}, \emph{cellsz}, \emph{scratchWorkspace}, \emph{version=None}}{}
Hydro-enforce a DEM

This aml is used by the National StreamStats Team as the optimal
approach for preparing a state’s physiographic datasets for watershed delineations.
It takes as input, a 10-meter (or 30-foot) DEM, and enforces this data to recognize
NHD hydrography as truth.  WBD can also be recognized as truth if avaialable for a
given state/region. This aml assumes that the DEM has first been projected to a
state’s projection of choice. This aml prepares data to be used in the Archydro
data model (the GIS database environment for National StreamStats).

The specified \textless{}8-digit HUC\textgreater{} should have a path associated with it in the variable 
settings section near the top of this aml.  The value entered will create a workspace
with this HUC id as it’s name, and copy all output datasets into the new workspace.
If the workspace already Exists, it should be empty before running this aml.

The snap\_grid is used to orient the origin coordinate of the output grids to align 
with neighboring HUC grids that have already been processed.  
Typically, this value is rounded to the nearest value
divisible by 10 (in cases where datsets are in units meters) or 30 (in cases where
datasets are in units feet).  The snap grid could be your input dem, if that grid
has already been rounded out (if topogrid was used and steps were followed on the 
nhd web page referenced above, then the input dem could be used).
\begin{quote}\begin{description}
\item[{Parameters}] \leavevmode\begin{description}
\item[{\sphinxstylestrong{outdir}}] \leavevmode{[}DEworkspace{]}
Working directory

\item[{\sphinxstylestrong{huc8cov}}] \leavevmode{[}DEFeatureClass{]}
Local division feature class, often HUC8, this will be the outer wall of the hydroDEM.

\item[{\sphinxstylestrong{origdemPth}}] \leavevmode{[}str{]}
Path to the orignial, projected DEM.

\item[{\sphinxstylestrong{dendrite}}] \leavevmode{[}str{]}
Path to the dendrite feature class to be used.

\item[{\sphinxstylestrong{snap\_grid}}] \leavevmode{[}str{]}
Path to a raster dataset to use as a snap\_grid to align all the watersheds, often the same as the DEM.

\item[{\sphinxstylestrong{bowl\_polys}}] \leavevmode{[}str{]}
Path to the bowling area raster generated from the bathymetric gradient tool.

\item[{\sphinxstylestrong{bowl\_lines}}] \leavevmode{[}str{]}
Path to the bowling line raster generated from the bathymetric gradient tool.

\item[{\sphinxstylestrong{inwall}}] \leavevmode{[}str{]}
Path to the feature class to be used for inwalling

\item[{\sphinxstylestrong{drainplug :}}] \leavevmode
Path to the feature class used for inserting sinks into the dataset

\item[{\sphinxstylestrong{buffdist}}] \leavevmode{[}float{]}
Distance to buffer the outer wall, same units as the projection.

\item[{\sphinxstylestrong{inwallbuffdist :}}] \leavevmode
Distance to buffer the inner walls, same units as the projection.

\item[{\sphinxstylestrong{inwallht :}}] \leavevmode
Inwall height, same units as the projection.

\item[{\sphinxstylestrong{outwallht :}}] \leavevmode
Inwall height, same units as the projection.

\item[{\sphinxstylestrong{agreebuf :}}] \leavevmode
AGREE function buffer distance, same units as the projection.

\item[{\sphinxstylestrong{agreesmooth :}}] \leavevmode
AGREE function smoothing distance, same units as the projection.

\item[{\sphinxstylestrong{agreesharp :}}] \leavevmode
AGREE function sharp distance, same units as the projection.

\item[{\sphinxstylestrong{bowldepth :}}] \leavevmode
Bowling depth, same units as the projection.

\item[{\sphinxstylestrong{cellsz :}}] \leavevmode
Cell size, same units as the projection.

\item[{\sphinxstylestrong{scratchWorkspace}}] \leavevmode{[}str{]}
Path to scratch workspace

\item[{\sphinxstylestrong{version}}] \leavevmode{[}str{]}
Package version number

\item[{\sphinxstylestrong{Returns (saved to outDIR)}}] \leavevmode
\item[{\sphinxstylestrong{——-}}] \leavevmode
\item[{\sphinxstylestrong{filldem}}] \leavevmode{[}raster{]}
hydro-enforced DEM raster grid saved to outDir

\item[{\sphinxstylestrong{fdirg}}] \leavevmode{[}raster{]}
HydroDEM FDR raster grid saved to outDir

\item[{\sphinxstylestrong{faccg}}] \leavevmode{[}raster{]}
HydroDEM FAC raster grid saved to outDir

\item[{\sphinxstylestrong{sink\_path}}] \leavevmode{[}feature class{]}
Sink feature class saved to outDir

\end{description}

\end{description}\end{quote}

\end{fulllineitems}

\index{moveRasters() (in module make\_hydrodem)@\spxentry{moveRasters()}\spxextra{in module make\_hydrodem}}

\begin{fulllineitems}
\phantomsection\label{\detokenize{make_hydrodem:make_hydrodem.moveRasters}}\pysiglinewithargsret{\sphinxcode{\sphinxupquote{make\_hydrodem.}}\sphinxbfcode{\sphinxupquote{moveRasters}}}{\emph{source}, \emph{dest}, \emph{rasters}, \emph{fmt=None}}{}
Move raster out of a working geodatabase to a destination folder.

\end{fulllineitems}

\index{postHydroDEM() (in module make\_hydrodem)@\spxentry{postHydroDEM()}\spxextra{in module make\_hydrodem}}

\begin{fulllineitems}
\phantomsection\label{\detokenize{make_hydrodem:make_hydrodem.postHydroDEM}}\pysiglinewithargsret{\sphinxcode{\sphinxupquote{make\_hydrodem.}}\sphinxbfcode{\sphinxupquote{postHydroDEM}}}{\emph{workspace}, \emph{facPth}, \emph{fdrPth}, \emph{thresh1}, \emph{thresh2}, \emph{sinksPth=None}, \emph{version=None}}{}
generate stream reaches, adjoint catchments, and drainage points
\begin{quote}\begin{description}
\item[{Parameters}] \leavevmode\begin{description}
\item[{\sphinxstylestrong{workspace}}] \leavevmode{[}str{]}
database-type workspace to output rasters and feature classes.

\item[{\sphinxstylestrong{facPth}}] \leavevmode{[}str{]}
Path to the flow accumulation grid produced by hydroDEM.

\item[{\sphinxstylestrong{fdrPth}}] \leavevmode{[}str{]}
Path to the flow direction grid produced by hydroDEM.

\item[{\sphinxstylestrong{thresh1}}] \leavevmode{[}int{]}
Threshold used to produce the str grid, in raster cells, usually equal to 15,000,000 m\$\textasciicircum{}2\$.

\item[{\sphinxstylestrong{thresh2}}] \leavevmode{[}int{]}
Threshold used to produce the str900 grid, in raster cells, usually equal to 810,000 m\$\textasciicircum{}2\$.

\item[{\sphinxstylestrong{sinksPth}}] \leavevmode{[}str (optional){]}
Path to the snklnk grid, optional.

\item[{\sphinxstylestrong{version}}] \leavevmode{[}str (optional){]}
StreamStats DataPrepTools version to be printed.

\end{description}

\item[{Returns}] \leavevmode\begin{description}
\item[{None}] \leavevmode
\end{description}

\end{description}\end{quote}

\end{fulllineitems}



\subsection{topo\_grid Module}
\label{\detokenize{topo_grid:module-topo_grid}}\label{\detokenize{topo_grid:topo-grid-module}}\label{\detokenize{topo_grid::doc}}\index{topo\_grid (module)@\spxentry{topo\_grid}\spxextra{module}}\index{topogrid() (in module topo\_grid)@\spxentry{topogrid()}\spxextra{in module topo\_grid}}

\begin{fulllineitems}
\phantomsection\label{\detokenize{topo_grid:topo_grid.topogrid}}\pysiglinewithargsret{\sphinxcode{\sphinxupquote{topo\_grid.}}\sphinxbfcode{\sphinxupquote{topogrid}}}{\emph{workspace}, \emph{huc8}, \emph{buffdist}, \emph{dendrite}, \emph{dem}, \emph{cellSize}, \emph{vipPer}, \emph{snapgrid=None}, \emph{huc12=None}}{}~\begin{quote}\begin{description}
\item[{Parameters}] \leavevmode\begin{description}
\item[{\sphinxstylestrong{workspace}}] \leavevmode{[}str{]}
Path to geodatabase

\item[{\sphinxstylestrong{huc8}}] \leavevmode{[}str{]}
Path to huc8 feature class

\item[{\sphinxstylestrong{buffdist}}] \leavevmode{[}int{]}
Distance to buffer huc8 in horizontal map units

\item[{\sphinxstylestrong{dendrite}}] \leavevmode{[}str{]}
Path to flowline dendrite feature class

\item[{\sphinxstylestrong{dem}}] \leavevmode{[}str{]}
Path to buffered, scalled, and projected DEM

\item[{\sphinxstylestrong{cellSize}}] \leavevmode{[}int{]}
Output cell size

\item[{\sphinxstylestrong{vipPer}}] \leavevmode{[}int{]}
VIP thining value

\item[{\sphinxstylestrong{snapgrid}}] \leavevmode{[}str (optional){]}
Path to snapgrid to use instead of input DEM.

\item[{\sphinxstylestrong{huc12}}] \leavevmode{[}list{]}
List of paths to HUC12 values if the huc8 doesn’t work

\end{description}

\item[{Returns}] \leavevmode\begin{description}
\item[{None}] \leavevmode
\end{description}

\end{description}\end{quote}
\subsubsection*{Notes}

See \sphinxurl{https://support.esri.com/en/technical-article/000004588}

\end{fulllineitems}



\chapter{Indices and tables}
\label{\detokenize{index:indices-and-tables}}\begin{itemize}
\item {} 
\DUrole{xref,std,std-ref}{genindex}

\item {} 
\DUrole{xref,std,std-ref}{modindex}

\item {} 
\DUrole{xref,std,std-ref}{search}

\end{itemize}


\renewcommand{\indexname}{Python Module Index}
\begin{sphinxtheindex}
\let\bigletter\sphinxstyleindexlettergroup
\bigletter{d}
\item\relax\sphinxstyleindexentry{databaseSetup}\sphinxstyleindexpageref{databaseSetup:\detokenize{module-databaseSetup}}
\indexspace
\bigletter{e}
\item\relax\sphinxstyleindexentry{elevationTools}\sphinxstyleindexpageref{elevationTools:\detokenize{module-elevationTools}}
\indexspace
\bigletter{m}
\item\relax\sphinxstyleindexentry{make\_hydrodem}\sphinxstyleindexpageref{make_hydrodem:\detokenize{module-make_hydrodem}}
\indexspace
\bigletter{t}
\item\relax\sphinxstyleindexentry{topo\_grid}\sphinxstyleindexpageref{topo_grid:\detokenize{module-topo_grid}}
\end{sphinxtheindex}

\renewcommand{\indexname}{Index}
\printindex
\end{document}